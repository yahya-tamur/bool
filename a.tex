\documentclass{article}

\usepackage{amsmath}
\usepackage{amssymb}
\usepackage{amsfonts}
% \usepackage{fourier}

\renewcommand{\thesubsection}{\thesection.\alph{subsection}}

\title{}
\author{}
\date{}

\begin{document}
  \maketitle
  \section{}

    Let $B$ be a ring where $x = x^2$ for all $x$. Some immediate consequences
    are:

    $B$ is commutative:

    Every element is its own additive inverse:

    \[ x + x = (x + x)^2 = x^2 + x^2 + x^2 + x^2 = x + x + x + x\]
    \[0 = x + x\]

    Multiplication is commutative:

    \[ x + y = (x + y)^2 = x^2 + xy + yx + y^2 = x + xy + yx + y\]
    \[0 = xy + yx\]
    \[xy = -yx = yx\]

    Where the last identity follows from our previous result.

    Now, define the following binary relation: $x \leq y$ iff $xy = x$. We get:

    \[xx = x \text{, so } x \leq x\]
    \[xy = x, yz = y \rightarrow xz = x(yz) = (xy)z = yz = x \text{, so } x
    \leq y, y \leq z \rightarrow x \leq z\]

    \[xy = x, yx = y \rightarrow x = xy = yx = y \text{, so } x \leq y, y \leq
    x \rightarrow x = y\]

    So, this is a reflexive partial order. Furthermore,

    \[0x = 0, x1 = x \rightarrow 0 \leq x, x \leq 1\]

    So we have greatest and least elements. We also have least upper bounds and
    greatest upper bounds:

    If $c \leq x, c \leq y$, then $c \leq xy \leq x, y$: By definition,
    $cx = c, cy = c$. Then,

    \[ cxy = cy = c \rightarrow c \leq xy\]
    \[ x(xy) = xy \rightarrow xy \leq x\]
    \[ y(xy) = (xy)y = xy \rightarrow xy \leq y\]


    If $x \leq c, y \leq c$, $x, y \leq x + xy + y \leq c$: By definition,
    $xc = x, yc = y$. Then,

    \[(x + y + xy)c = xc + yc + xyc = x + y + xy \rightarrow x + y + xy \leq c\]
    \[(x + y + xy)x = x + yx + yx = x \rightarrow x \leq x + y + xy\]
    \[(x + y + xy)y = xy + y + xy = y \rightarrow y \leq x + y + xy\]

    Denoting the operations $x \frown y = xy, x \smile y = x + y + xy$, these
    operations a

    This is object is called a boolean algebra. We will now proceed to define
    them by starting from a lattice and defining a ring structure. One advantage
    of that approach is that in order to verify a map between boolean algebras
    is a homomorphism, it will suffice to check that it respects the order
    relation ($x \leq y$ iff $f(x) \leq f(y)$). We will also need the definition
    based on rings to discuss ideals later.

    \section{}

    A partial order with least upper bounds ('joins') and greatest lower bounds
    ('meets') is called a lattice; a lattice with a greatest and least element
    is called a bounded latice. A lattice where $\smile$ and $\frown$ distribute
    over each other is called a a distributive lattice, and a lattive where, for
    every $x$ there's a $x'$ with $x \smile x' = 1$, $x \frown x' = 0$ is called
    a complemented lattice. Such a complement is unique -- let $x'$, $x''$ be
    complements of $x$. Then,

    \[x' \smile x'' = (x' \smile  x'') \frown 1\]
    \[= (x' \smile  x'') \frown (x \smile x'')\]
    \[= (x' \frown x) \smile x''\]
    \[= 0 \smile x'' = x''\]

    So, $x' \leq x''$, by applying the definition in reverse we get $x'' \leq
    x'$ so $x' \ x''$. In particular, this shows that the complement of the
    complement of $x$ is $x$.

    We can also use this to prove de Morgan's laws:

    \[(x' \smile y') \frown (x \frown y) = (x' \frown (x \frown y)) \smile (y'
    \frown (x \frown y))\]
    \[ = (0 \frown y) \smile (0 \frown x) = 0\]

    \[(x' \smile y') \smile (x \frown y) = ((x' \smile y') \smile x) \frown (x'
    \smile y') \smile y)\]
    \[ = (1 \smile y') \frown (1 \smile x') = 1\]

    So, $x' \smile y'$ is the complement of $x \frown y$:
    \[(x' \smile y') = (x \frown y)'\]
    Replacing $x$ and $y'$ by their respective complements, and taking the
    complement of both sides,
    \[(x'' \smile y'')' = (x' \frown y')''\]
    \[(x \smile y)' = x' \frown y'\]

    To make this a ring, the multiplication is defined as $\frown$, and the
    addition is defined as:

    \[ x + y = (x \frown y') \smile (x' \frown y)\]

    I'll omit the details, but this does form a ring, with $x^2 = x \frown x =
    x$, so a boolean algebra.

    Our previous definition of $\smile$ is compatible:


    \[A = \{X: X \subseteq \mathbb{N}: X \text{ or } \mathbb{N} - X \text{ is
    finite} \} \]

    \section{Homomorphisms and Isomorphisms}

    A homomorphism of boolean algebras is a homomorphism of rings that happen to
    be boolean albegras -- maps $h$ so that

    \[h(x+y) = h(x) + h(y)\]
    \[h(xy) = h(x)h(y)\]
    \[h(1) = 1\]

    The condition $h(0)= 0$ can be derived from:

    \[h(0) = h(0+0) = h(0)+h(0) = 0\]

    From this definition, we can easily prove

    \[h(x \frown y) = h(xy) = h(x)h(y) = h(x) \frown h(y)\]
    \[h(x') = h(x+1) = h(x) + h(1) = h(x) + 1 = h(x)'\]
    \[h(x \smile y) = h(x + y + xy) = h(x) + h(y) + h(x)h(y) = h(x) \smile
    h(y)\]
    \[x \leq y \rightarrow xy = x \rightarrow h(xy) = h(x) \rightarrow h(x) \leq
    h(y)\]

    It also suffices to specify

    \[ h(x \frown y) = h(x) \frown h(y) \]
    \[h(x') = h(x)'\]

    Since
    \[x + y = (x \frown y') \smile (x' \frown y) = (x \frown y')' \frown (x'
    \frown y)'\]

    An isomorphism of Boolean algebras is a bijective homomorphism. An inverse
    of an isomorphism is also a homomorhpism:




    \section{Ideals}

      In the following, 'ideal' means proper ideal, or an ideal which is not the
      whole ring. For a subset $I$ of a Boolean algebra $A$ to be an ideal, the
      following are necessary and sufficient:

      \[0 \in I, 1 \notin I\]
      \[\text{for all } x, y \in I, x \smile y \in I\]
      \[\text{for all } x \in I, y \in A, y \leq x \text{ implies } y \in I\]

      % special formatting?
      Proof:
      
      Let $I$ be an ideal. $0$ is in $I$, since it's an additive subgroup. $1$
      isn't, since then, for every $a \in A$, $a*1 = a$ is in $I$, then $I$
      would have to be the entire ring.

      If $x$ and $y$ are in $I$, by definition, $xy$ is too. Then, since $I$ is
      an additive subgroup, $x \smile y = x + y + xy$ is as well.

      Finally, if $x$ is in $I$, $xy \in I$. $y \leq x$, by definition means
      $xy = y \in I$.

      Conversely, let these conditions be satisfied.

      If $x, y \in I$, $x \smile y \in I$.
      \[(x+y)(x \smile y) = (x+y)((x+y)+xy) = x + y + (x+y)xy = x + y + xy + xy =
      x + y\]
      So, $x+y \leq x \smile y$, and $x+y \in I$ by the third condition. Since
      $-x = x$ in a boolean algebra, we conclude that $I$ is an additive
      subgroup.

      For $x \in I$, $xy = x \frown y \leq x$, so $xy \in I$. So, $I$ is an
      ideal. It's proper, since it doesn't include the element $1$.

      The equivalent conditions define what's called an ideal in the context of
      partially ordered sets. We've proven that they're equivalent to ideals for
      rings in Boolean algebras.

      Then, the following are equivalent:

      \[(1) \text{I is a maximal ideal}\]
      \[(2) A/I \text{ is isomorphic to } \{0,1\}\]
      \[(3) \text{I is the kernel of a homomorphism } A \rightarrow \{0,1\}\]
      \[(4) \text{For all } x, x \in I \text{ or } x+1 \in I\]
      \[(5) \text{For all } x,y, \text{ if } xy \in I, \text{ then } x \in I
      \text{ or } y \in I\]
      \[(6) \text{For all } x_1x_2...x_k \in I, x_1 \in I \text{ or } x_2 \in I
      \text{ or } ... x_k \in I\]

      $(1) \rightarrow (2)$

      In fact, in any commutative ring, we can prove that $A/I$ is a field if
      and only if $I$ is maximal:

      If $I$ isn't maximal, let $I \subsetneq J$, and $a \in J - I$. $a$ isn't
      in $I$, so $[a]$ isn't the zero element in $A/I$. If it was invertible,
      there would be a $b$ with
      \[[a][b] = 1\]
      \[[ab - 1] = 0\]
      \[ab - 1 \in I\]
      $a \in I$, so $ab \in I$. Then,
      \[ab - (ab - 1) = 1 \in I\]
      This is a contradiction.

      Conversely, assume $I$ is maximal. Let $[a] \neq 0 \in A/I$. Form
      \[K = \{ay + z: y \in A, z \in I\} \]
      Clearly, $0 = a*0 + 0 \in K$, if $ay_1 + z_1$, $ay_2 + z_2 \in K$, since
      the ring is commutative,
      \[ay_1 + z_1 -(ay_2 + z_2) = a(y_1-y_2) + (z_1 - z_2) \in K\]
      And, if $ay + z \in K$,
      \[(ay + z)t = a(yt) + (zt) \in K\]
      So, $K$ is an ideal. It includes any element $a*0 + y$ of $I$, and it
      includes $a*1 + 0 = a$, which isn't in $I$. Since $I$ is maximal, $K$ must
      be the whole ring, and then, there must be $y, z$ with
      \[ay + z = 1\]
      Then,
      \[ay + z= [a][y] + [z] = [a][y] = 1\]
      Note that $z \in I$. Then, any nonzero element of $A/I$ does have an
      inverse, and this proof is complete.

      Finally, note that $\{0,1\}$ is the only Boolean algebra which is a field:
      For every $x$,
      \[(x+1)x = x + x = 0\]
      Which, if this is a field (or even an integral domain), implies that
      either $x=0$ or $x=1$.

      $(2) \rightarrow (3)$ $I$ is the kernel of the quotient map $A \rightarrow
      A/I = \{0,1\}$.

      $(3) \rightarrow (4)$ Let $h$ be the homomorphism $A \rightarrow \{0,1\}$
      with kernel $I$. Then, for all $x$, $h(x) = 0$ or $h(x) = 1$. In the
      latter case, $h(1+x) = h(1) + h(x) = 1 + 1 = 0$. So, $x$ or $1+x$ is in
      the kernel, which is $I$.

      $(4) \rightarrow (5)$ Let $x, y \notin I$. Then, $1+x, 1+y \in I$. Then,
      \[(1+x) \smile (1+y) = 1 + (x \frown y) = 1 + xy \in I\]
      So, $xy \in I$ implies $x \in I$ or $y \in I$.

      $(5) \rightarrow (1)$ Suppose $I$ isn't maximal. Then, let $I \subsetneq
      J$ be an ideal, $a \in J - I$. $a \in J$, so $1+a \notin J$, because then
      \[a \smile (1+a) = 1 \in J\]
      But, $(1+a) \in I \subseteq J$, which is a contradiction.

      $(5) \leftrightarrow (6)$ This is an obvious use of induction.

    \section{Filters}

      Filters in order theory are the dual notion of ideals:

      A filter is a subset $F$ of a Boolean algebra $A$ such that:

      \[0 \notin F, 1 \in F\]
      \[x, y \in F \rightarrow x \frown y \in F\]
      \[x \in F, y \geq x \rightarrow y \in F\]

      In boolean
      algebras, we can equivalently define them as subsets $F$ such that $\{x
      \in A: 1+x \in F\}$ is a filter.

      Proof:

      Let
      \[I = \{x \in A: 1+x \in F\}\]


      Assume the first set of conditions hold.

      The first condition implies $0 \in I, 1 \notin I$.

      The second condition implies for $x, y \in I$, $(1+x), (1+y) \in F$, then
      \[(1+x) \frown (1+y) \in F\]
      \[1 + (x \smile y) \in F\]
      \[x \smile y \in I\]
      Finally, notice that $x \leq y$ imples $1+y \leq 1+x$:
      \[xy = x\]
      \[(1+x)(1+y) = 1 + x + y + xy = 1 + x + y + x = 1 + y\]
      Then, if $x \in I$, $y \leq x$, $1+x \in F$, $1+y \geq 1+x$, so by the
      third condition, $1+y \in F$, and $y \in I$.
      So, $I$ is an ideal as we have defined.

      Conversely, if $I$ is an ideal, the proof is entirely the same:
      
      The first condition is given by $0 \in I, 1 \notin I$.

      If $x, y \in F$, $1+x, 1+y \in I$, then
      \[(1+x) \smile (1+y) \in I\]
      \[1 + (x \frown y) \in F\]
      \[x \frown y \ in F\]

      And finally, if $x \in F$, $y \geq x$,
      \[1+x \in I, 1+y \leq 1+x \rightarrow 1+y \in I\]
      \[y \in F\]

      We also have a similar characterization for maximal filters, which are
      called ultrafilters. The following are equivalent:

      \[(1') F \text{ is an ultrafilter}\]
      \[(3') \text{ there's a homomorphism } g: A \rightarrow \{0,1\} \text{ with }
      F = h^{-1}(\{1\})\]
      \[(4') \text{ for all x}, x \in F \text{ or } 1+x \in F\]
      \[(5') \text{ if } x \smile \in F, \text{ then } x \in F \text{ or } y \in
      F\]
      \[(6') \text{ if } x_1 \smile ... \smile x_n \in F, \text{ then } x_1 \in
      F \text{ or } ... x_n \in F\]

      We'll prove this be proving that if and only if $F$ has one of these
      properties, the dual filter $I$ as defined above has the corresponding
      property.

      $(1')$ Saying there's no filter $F' \supsetneq F$ is equivalent to saying
      there's no ideal $I' \supsetneq I$, since a set is a filter if and only if
      the set of its complements is an ideal and vice versa.

      $(3')$ For a boolean algebra homomorphism, $h(x+1) = h(x) + 1$, so a
      homomorphism sending $F$ to 1 is equivalent to a homomorphism sending $I$
      to 0.

      $(4')$ This is obvious by the definition of $I$ -- $x \in I$ if and only
      if $x+1 \in F$, etc.

      $(5')$ If this condition is true, for $x \frown y \in I$, $1+ x \frown y$
      = $(1+x) \smile (1+y) = $ is in $F$, so $1+x$ or $1+y$ is in $F$, so $x$
      or $y$ is in $I$.

      Similarly, if the corresponding statement is true for $I$, let $x \smile y
      \in F$.
      \[I \ni 1 + (x \smile y) = (1 + x) \frown (1 + y)\]
      \[1 + x \in I \text{ or } 1 + y \ in I\]
      And $x$ or $y$ is in $F$.

      Finally, $(6')$ may be shown to be equivalent to $(5')$ by induction.

      One fact that will be especially impactful later is that we've proven in
      properties $(3)$ and $(3')$ that every homomorphism $A \rightarrow
      \{0,1\}$ corresponds bijectively to a maximal ideal and an ultrafilter.

      Remark.

      For any $x \neq 0 \in A$, we can form a filter:
      \[ F = \{y \in A: y \geq x\}\]

      It's easy to see that $0 \notin F$, $1 \in F$, $u, v \in F$ implies $u
      \frown v \in F$, and $u \in F, v \geq u$ implies $v \in F$, so it's indeed
      a filter.

      Krull's theorem, which is proven with the axiom of choice, states that
      every ideal in a commutative ring is included in some maximal ideal. This
      implies that ecery element in $x$ is included in some ultrafilter.
%      \subsection{Filterbases}
%
%        skipped for now -- I'd be copying the book too closely. we'll see what
%        we need later.





    \section{Topological Spaces}

      \subsection{An equivalent condition for compactness}

        %%%%Probably not necessary: {
        %It will be convenient later to rephrase the definition of compactness as
        %follows.
%
        %The usual definition states that every open cover of a topological space
        %has a finite subcover. We'll also assume the topological space has to be
        %Hausdorff. Consider the following condition:
%
        %Every open cover of a topological space by sets in some basis has a
        %finite subcover.
%
        %Clearly the usual definition implies this; conversely, if this is true,
        %for any open cover $\{U_i\}_{i \in I}$, we can form an open cover by
        %basis elements by finding a cover $\{V_{ij}\}_{j \in J_i}$ of every
        %$U_i$ by basis elements, find a finite subcover
        %$\{V_{i_kj_k}\}_{k=1}^n$ and then pick the open set these basic sets
        %were in in the original cover:
        %\[\{U_{i_k}\}_{k=1}^n\]
        %This covers the topological space, since $\{V_{i_kj_k}\}_{k=1}^n$ did as
        %well.
%
        %Finally, we can state this condition by equivalently talking about the
        %complements:
%
        %For every set of (closed) complements of basic open sets whose total
        %intersection is empty, we can find a finite subset whose intersection is
        %still empty.
%
        %%%%%}
%
      \subsection{Zero dimensional topological spaces}

        A topological space is said to be zero dimensional if it has a basis
        consisting of closed sets. We won't get into a definition of dimension
        in general for topological spaces here.

        We can equivalently say that the family of sets which are both open and
        closed ('clopen') form a basis for the topology. This clearly implies
        the previous condition, and if the previous statement is true, the basis
        consisting of closed sets is a subset of the set of clopen sets, so the
        latter forms a basis as well.

        A compact zero dimensional topological space is called a Boolean
        topological space.

        Given an indexed family of topological spaces $(X_i)_{i \in I}$, we can
        define a topology on their product $\Pi_{i \in I} X_i$ -- elements
        $\Pi_{i \in I} O_i$, where all the $O_i$ are open in $X_i$ and all but
        finitely many ones are equal to $X_i$. Tychonoff's Theorem, which is
        equivalent to the axoim of choice, states that if all the $X_i$ are
        compact, this product space is too.

        Take the discrete topology on two points $\{0,1\}$,
        and form the product space $\{0,1\}^I$ for some indexing set $I$. A
        basic open set $\Omega$ is either empty (in case when one of the open
        sets is the empty set), or has a finite number of indices which are
        either $\{0\}$ or $\{1\}$. Where the element  of the product space are
        written as functions $I \rightarrow \{0,1\}$, it's the set of
        functions whose values at a finite number of points is determined:

        \[\{f: I \rightarrow \{0,1\}: f(i_1) = \epsilon_1, ... f(i_n) =
        \epsilon_n\}\]

        Where $n$ is some natural number and each $\epsilon_i$ is either $0$ or
        $1$. By De Morgan's Law, the complement of this set is

        \[\bigcup_{1 \leq j \leq n} \{f : f(i_j) = 1 - \epsilon_j\}\]

        This is a finite union of open sets in the basis, so it's open. Then,
        $\Omega$ is closed as well. Since every set in the basis is clopen, this
        is a zero dimensional topological set. $\{0,1\}$ is clearly compact. By
        Tychonoff's Theorem, this space is compact as well, and therefore a
        Boolean topological space.

    \subsection{The Stone space of a Boolean algebra}

      For a Boolean algebra $A$, the subset of $\{0,1\}^A$ which consists of
      homomorphisms $A \rightarrow \{0,1\}$ is denoted $S(A)$ and called the
      Stone space of $A$.

      We've seen that $\{0,1\}^A$ is zero dimensional, so we can find a basis by
      clopen sets. With $S(A)$ under the subspace topology, the same basis forms
      a basis of clopen sets, so $S(A)$ is zero dimensional as well.

      Now, we'll characterize the basis of $S(A)$. Note that we've characterized
      the basis of $\{0,1\}^A$ as functions which take a finite set of points to
      either $0$ or $1$.

      The sets in the basis of $S(A)$ are exactly the sets defined as
      \[\{h\in S(A): h(a) = 1\}\]
      for some $a \in A$. This element $a$ is unique.

      Clearly, every set defined this way is a set in the basis -- $S(A) \cap
      \{f: A \rightarrow \{0,1\} : f(a) = 1\}$.

      Every set in the basis can be defined this way:

      Let $\Delta$ be an arbitrary element in the basis, given by
      \[\Delta = \{h \in S(A) : h(a_1) = \epsilon_1, ... h(a_n) = \epsilon_n\}\]

      Where $a_i \in A$, and $\epsilon_i$ are either zero or one. Define:
      \[b_k = a_k \text{ if } \epsilon_k = 1, 1+a_k \text{ otherwise }\]
      Since the elements of $S(A)$ are homomorphisms, if $\epsilon_k = 1$,
      \[h(b_k) = h(a_k)\]
      and, if $\epsilon_k = 0$,
      \[h(b_k) = h(1+a_k) = 1+h(a_k)\]

      In other words, $h(b_k) = 1$ if and only if $h(a_k) = \epsilon_k$. Then,
      $h(a_1) = \epsilon_1, h(a_2) = \epsilon_2, ... h(a_n) = \epsilon_n$
      if and only if
      \[h(b_1) \frown h(b_2) \frown ... \frown h(b_n) = 1\]
      Since $h$ also preserves intersections, this happens if and only if
      \[h(b_1 \frown b_2 \frown ... \frown b_n) = 1\]

      As maybe a special case, notice that the empty set and $S(A)$ are given
      by $\{h \in S(A) : h(0) = 1\}$ and $\{h \in S(A) : h(1) = 1\}$
      respectively.

      For uniqueness, let $a \neq b$. Then, $a+b \neq 0$.
      Then, there's an ultrafilter that includes $a+b$, so a homomorphism with
      
      \[\phi(a+b) = 1 = \phi(a) + \phi(b)\]

      Since the right hand side is in $\{0,1\}$, this implies $\phi(a) = 1$ and
      $\phi(b) = 0$, or vice versa. Without loss of generality, in this case,

      \[\phi \in \{h \in S(A): h(a) = 1\}\]
      \[\phi \notin \{h \in S(A) : h(b) = 1\}\]

      So the sets aren't equal.

      As a corollary, the complements of the sets in the basis are exactly the
      sets in the basis:

      \[\{h \in S(A) : h(a) = 1\}^c = \{h \in S(A) : h(1+a) = 1\}\]

      Finally, here's a proof that the $S(A)$ is closed. Remember that a
      function between Boolean algebras is a homomorphism if and only if it
      preserves intersections and complements.

      Let
      \[\Omega(a,b) = \{ f: A \rightarrow \{0,1\} : f(ab) = f(a)f(b), f(1+a) = 1
      + f(a)\}\]

      Then, by definition,

      \[S(A) = \bigcap_{a \in A, b \in A} \Omega(a,b)\]

      Finally, we can write $\Omega(a,b)$ as the following union of closed sets:

      \[\Omega(a,b) = \{f : f(a) = 0, f(b) = 0, f(ab) = 0), f(1+a) = 1\}
      \cup \{f : f(a) = 0, f(b) = 1, f(ab) = 0), f(1+a) = 1\}
      \cup \{f : f(a) = 1, f(b) = 1, f(ab) = 0), f(1+a) = 0\}
      \cup \{f : f(a) = 1, f(b) = 1, f(ab) = 1), f(1+a) = 0\}
      \]

      So, $S(A)$ is some arbitrary intersection of closed sets, and is therefore
      closed itself. As a closed subset of a compact space, it's compact. Then,
      since it's also zero-dimensional, it's a Boolean topological space.

      We can also show that the clopen sets are exactly the sets in this basis.

      Every basis set we've found is already both open and closed. Given an open
      and closed set $\Gamma$,

      Since it's open, it's covered by sets in the basis $\{\Gamma_i\}_{i \in I}$
      Since $\Gamma$ is a closed subset of a compact space, it's compact. Then,
      find a finite subcover $\Gamma_1, ... \Gamma_n$. Each has form:
      \[\Gamma_i = \{h: h(x_i) = 1\}\]
      Then,
      \[\Gamma = \{h: h(x_1 \smile x_2 \smile ... \smile x_n)\}\]

    \section{}

      The Boolean algebra of clopen subsets of a topological space $S$ will be
      denoted by $B(S)$. This is a Boolean algebra with regular set operations
      union, intersection, complement by $S$, and is clearly closed under all of
      these.

      We'll now prove that every Boolean algebra is isomorphic to the Boolean
      algebra of clopen sets of its Stone space.

      Let $H$ be the map $A \rightarrow \mathcal{P}(S(A))$

      \[H(a) = \{h \in S(A) : h(a) = 1\}\]

      We've done a lot of the work in the previous section: We know that $H(a)$
      is a clopen subset of $S(A)$ for any $a$, and that any clopen subset is in
      the image, so the map is surjective onto $B(S(A))$. We also know that this
      map is inject      \[= h \mapsto (a \mapsto h ( H_{A'}^{-1} ( \alpha^{-1} (H_A (a))))) \]
      
      \[= h \mapsto (a \mapsto h ( A_{A'}^{-1} ( \alpha^{-1} \{h \in S(A) :
      h(a) = 1 \} )))\]

      \[= h \mapsto (a \mapsto h ( A_{A'}^{-1} ( \{h' \in S(A') :
      \alpha(h') \in \{ h \in S(A) : h(a) = 1\} \} )))\]

      \[= h \mapsto (a \mapsto h ( A_{A'}^{-1} ( \{h' \in S(A') :
      \alpha(h')(a) = 1\} )))\]
ive, since we've seen that $a$ is unique.

      To show that it's an isomorphism, we'll show that $H(a) \leq H(b)$ if and
      only if $a \leq b$.

      If $x \leq y$, and $h$ is any homomorphism with $h(x) = 1$, $h(x) \leq
      h(y)$ implies that $h(y) = 1$. Then, $H(x) \subseteq H(y)$.

      Now, assume $x$ is not less than or equal to $y$. Then, $xy \neq x$,
      $x(1+y) \neq 0$. Consider the ultrafilter which includes this element, and
      the homomorphism associated with it.
      \[h(x(1+y)) = 1\]
      So
      \[h(x) = 1, h(y) = 0\]
      And then, $h \in H(x)$, $h \notin  H(y)$, and $H(x) \nsubseteq H(y)$.


      We can also now prove that every Boolean topological space $X$ is
      homeomorphic to the Stone space of the Boolean algebra of clopen subsets of
      $X$.

      We've seen that the Boolean algebra of clopen sets of $X$ is a basis for
      the topology on any Boolean topological space. For all $x \in X$, define

      \[f_x: B(X) \rightarrow \{0,1\} = \Omega \mapsto \{1 \text {if} x \in
      \Omega, 0 \text {otherwise}\}\]

      First, we'll show that $f_x$ is really a homomorphism of Boolean algebras,
      so in $S(B(X))$. 

      For clopen subsets $\Omega, \Delta$,
      \[f_x(\Omega \cap \Delta) = 1 \text{ iff } x \in \Omega \cap \Delta\]
      \[= 1 \iff f_x(\Omega) = 1 \text{ and } f_x(\Delta) = 1\]
      \[= f_x(\Omega)f_x(\Delta)\]

      \[f_x(\Omega^c) = 1 \text{ iff } x \in \Omega^c\]
      \[ = 0 \text{ iff } x \in \Omega\]
      \[ = f_x(\Omega)^c\]

      So, $f_x$ is a Boolean algebra homomorphism from $B(X)$ onto $\{0,1\}$, so
      is in $S(B(X))$.

      Now, we'll show that it's injective.

      ]If $x \neq y$, because $X$ is Hausdorrf, we can find an open set $O$ with
      $x \in O$, $y \notin O$. $O$ is a union of basic open sets, so there's a
      clopen set $\Omega \in B(X)$ with $x \in \Omega$, $y\notin \Omega$. Then,
      \[f_x(\Omega) = 1 \neq f_y(\Omega) = 0\]
      So, $f_x \neq f_y$, and $f$ is injective.

      Now, we'll show that it's surjective.

      Let $h$ be an arbitrary element of $S(B(X))$, so a homomorphism $B(X)
      \rightarrow \{0,1\}$. The filter associated with $h$ is
      \[U = \{\Omega \in B(X) : h(\Omega) = 1\}\]

      Since $U$ is a filter, it has the property that if $x_1, ... x_n \in U$,
      their intersection $x_1 \frown x_2 \frown ... \frown x_n$ is in $U$, so
      it's nonzero -- in this case, it's not the empty set.

      The usual definition of compactness states that every open cover has a
      finite subcover. By taking complements of every set in the cover, this
      equivalently states that every infinite set of closed sets whose
      intersection is empty has a finite subset of closed sets whose
      intersection is still empty.

      Now, since $X$ is compact, and the elements of $U$ are closed, and every
      finite intersection of elements of $U$ are nonempty, we can conclude that
      the intersection of every element of $U$ is still nonempty.

      Let $x$ be an element of this intersection.

      For every clopen subset $\Omega$ of $B(X)$,

      $\Omega$ might be an element of $U$. In this case, since $c$ is in every
      element of $U$, it's in $\Omega$, and
      \[f_x(\Omega) = 1\]
      If $\Omega$ isn't an element of $U$, $1+\Omega$ is, since $U$ is an
      ultrafilter. Then, $x$ is in $1+\Omega$, and
      \[f_x(1+\Omega) = 1\]
      Then, since $f_x$ is a homomorphism,
      \[f_x(\Omega) = 0\]

      Then,
      \[f_x(\Omega) = 1 \text{ iff } \Omega \in U\]
      And, by definition, $U = \{\Omega \in B(X) : h(\Omega) = 1\}$. So,
      \[f_x = h\]
      Then, $H: f \mapsto f_x$ is surjective onto $S(B(X))$. Since we know that
      $f$ is injective, this also tells us the $x$, the intersection of all the
      elements of $U$, is unique -- $h = f_x = f_y$ implies $x=y$.

      Finally, we can prove that $f$ is continuous:

      Let $G$ be an element in the basis of $S(B(X))$, so a clopen set. We've
      seen that i Stone spaces, this means it has the form

      \[G = \{h \in S(B(X)): h(\Omega) = 1\}\]

      For some $\Omega \in B(X)$.

      \[f^{-1}(G) = \{x \in X : f_x \in G\}\]
      \[ = \{x \in X : f_x(\Omega) = 1\}\]
      \[ = \{x \in X : x \in \Omega\} = \Omega\]

      Where $\Omega \in B(X)$, so it's open. Since the inverse image of elements
      in the basis are open, $f$ is continuous.

      $f^{-1}$ is continuous:

      Let $\Omega \in X$ be an open set in the basis of $X$.

      \[(f^{-1})^{-1}(\Omega) = f(\Omega) = \{f_x \in S(B(X)) : x \in \Omega\}\]

      We've seen that $f$ is surjective, so every $h \in S(B(X))$ has form $f_x$
      for some $x$. Then, $h(\Omega) = 1$ if and only if $x$ such that $h = f_x$
      is in $\Omega$ -- by definition of $f$, $f_x(\Omega) = 1$ if and only if
      $x$ is in $\Omega$. This means:

      \[\{f_x \in S(B(X)) : x \in \Omega\} = \{h \in S(B(X)) : h(\Omega) = 1\}\]

      The latter is a basis element in the Stone space of $B(X)$, so we conclude
      that $f^{-1}$ is continuous as well.

      Name the isomorphism $A \to B(S(A))$ $H_A$.

    \section{Categorical equivalence}

      We will now construct a bijection between homomorphisms between Boolean
      algebras $A$ to $A'$ and continuous functions $S(A')$ to $S(A)$.

      Since elements of $S(A)$ can be regarded as functions $A \rightarrow
      \{0,1\}$, we can define this mapping similarly to dual spaces in vector
      spaces:

      \[\Phi(\phi) = (h : A' \rightarrow \{0,1\}) \mapsto (h \circ \phi : A
      \rightarrow \{0,1\})\]

      We'll first show that this map is actually continuous.

      Let $\Omega$ be an element in the basis of $S(A)$. It has form $\{h \in
      S(A): h(a) = 1\}$ for some $a \in A$.

      \[(\Phi(\phi))^{-1}(\Omega) = \left\{h' \in S(A') : \Phi(\phi)(h') \in \{h
      \in S(A): h(a) = 1\} \right\}\]
      \[ = \{h' \in S(A') : (\Phi(\phi)(h'))(a) = 1 \}\]
      \[ = \{h' \in S(A') : (h' \circ \phi)(a) = 1 \}\]
      \[ = \{h' \in S(A') : h'(\phi(a)) = 1\}\]

      This is an open set in $S(A')$, so this map is continuous.

      We'll show that this is a bijection by defining an inverse. If $\alpha$ is
      a continuous function $S(A')$ to $S(A)$, let $\alpha^{-1}$  be the
      function $\mathcal{P}(S(A)) \to \mathcal{P}(S(A'))$. Since $\alpha$ is
      continuous, $\alpha^{-1}$ takes clopen sets to clopen sets, and so it's a map
      $B(S(A')) \to B(S(A))$. Then, the following defines a function $A \to
      B(S(A)) \to B(S(A')) \to A'$:

      \[\Psi(\alpha) = H_{A'}^{-1} \circ \alpha^{-1} \circ H_A\]

      \[(\Psi \circ \Phi)(\phi) = \Psi(\Phi(\phi))\]
      \[ = H_{A'}^{-1} \circ (\Phi(\phi))^{-1} \circ H_A \]
      \[ = a \mapsto (H_{A'}^{-1} \circ (\Phi(\phi))^{-1})  \left(\{h \in S(A) : h(a) =
      1 \}\right) \]
      \[ = a \mapsto H_{A'}^{-1} \left( \left\{g \in S(A') : \Phi(\phi)(g) \in \{h \in S(A) : h(a) =
      1 \}\right\} \right) \]
      \[ = a \mapsto H_{A'}^{-1} (\{g \in S(A') : ((\Phi(\phi))(g))(a) = 1 \}) \]
      \[ = a \mapsto H_{A'}^{-1} (\{g \in S(A') : ((g \circ \phi)(a) = 1 \})) \]
      \[ = a \mapsto H_{A'}^{-1} (\{g \in S(A') : g(\phi(a)) = 1 \}) \]
      \[ = a \mapsto H_{A'}^{-1} (H_{A'}(\phi(a))) \]
      \[ = a \mapsto \phi(a)\]
      \[ = \phi\]

      \[(\Phi \circ \Psi)(\alpha) = \Phi(H_{A'}^{-1} \circ \alpha^{-1} \circ
      H_A)\]
      \[= h \mapsto h \circ H_{A'}^{-1} \circ \alpha^{-1} \circ H_A \]

      Let $U$ be the argument of $h \circ H_{A'}^{-1}$ in the expression, so

      \[U \in S(B(A)) = \alpha^{-1} (H_A (a))\]
      \[ = \alpha^{-1} \{h \in S(A) : h(a) = 1 \}\]
      \[ = \{h' \in S(A') :\alpha(h') \in \{ h \in S(A) : h(a) = 1\} \}\]
      \[ = \{h' \in S(A') : \alpha(h')(a) = 1\} \]


      $U$ is equal to $\{h \in S(X) : h(\Omega) = 1\}$ for some $\Omega \in
      B(X)$. Then,

      \[ h ( H_{A'}^{-1} (U)) =  h(H_A^{-1}(\{h \in S(X) : h(\Omega) = 1\})) = h(H_A^{-1}(H_A(\Omega))) =
      h(\Omega)\]

      By definition, $\Omega$ is the number such that $h \in U$ if and only if
      $h(\Omega) = 1$. Then,
      \[ h ( H_{A'}^{-1} (U)) = 1 \text{ if and only if } h \in U\]
      \[ h ( H_{A'}^{-1} (U)) = 1 \text{ if and only if } h \in \{h' \in S(A') :
      \alpha(h')(a) = 1\} \]
      \[ h ( H_{A'}^{-1} (U)) = 1 \text{ if and only if } \alpha(h)(a) = 1\]
      Since all the possible values are $0$ or $1$,

      \[ h ( H_{A'}^{-1} (U)) = \alpha(h)(a)\]
      \[(\Phi \circ \Psi)(\alpha) = h \mapsto (a \mapsto \alpha(h)(a))\]
      \[(\Phi \circ \Psi)(\alpha) =  h \mapsto \alpha(h)\]
      \[(\Phi \circ \Psi)(\alpha) = \alpha\]

    To Do:

      So, $\Phi$ defines a map from the morphisms of the category of Boolean
      algebras and homomorphisms to the morphisms of the category of Boolean
      topological spaces and continuous functions. With $S$ mapping the objects,
      it's a functor -- $\Phi(\text{id}) = \text{id}$ and $\Phi(g \circ f) =
      \Phi(g) \circ \Phi(f)$ is satisfied, since it's a very standard
      construction:




      -- section on homomorphisms + isomorphisms
      -- category equicalence
      -- organize better
      -- x + 1 for complements isn't great, introduce + use $x^c$ notation

      -- future content:
      -- equivalent forms of aoc
      -- atomless + countable boolean algebra

\end{document}


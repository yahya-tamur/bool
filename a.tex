\documentclass{article}

\usepackage{amsmath}
\usepackage{amsfonts}
% \usepackage{fourier}

\renewcommand{\thesubsection}{\thesection.\alph{subsection}}

\title{}
\author{}
\date{}

\begin{document}
  \maketitle
  \section{}

    Let $B$ be a ring where $x = x^2$ for all $x$. Some immediate consequences
    are:

    $B$ is commutative:

    Every element is its own additive inverse:

    \[ x + x = (x + x)^2 = x^2 + x^2 + x^2 + x^2 = x + x + x + x\]
    \[0 = x + x\]

    Multiplication is commutative:

    \[ x + y = (x + y)^2 = x^2 + xy + yx + y^2 = x + xy + yx + y\]
    \[0 = xy + yx\]
    \[xy = -yx = yx\]

    Where the last identity follows from our previous result.

    Now, define the following binary relation: $x \leq y$ iff $xy = x$. We get:

    \[xx = x \text{, so } x \leq x\]
    \[xy = x, yz = y \rightarrow xz = x(yz) = (xy)z = yz = x \text{, so } x
    \leq y, y \leq z \rightarrow x \leq z\]

    \[xy = x, yx = y \rightarrow x = xy = yx = y \text{, so } x \leq y, y \leq
    x \rightarrow x = y\]

    So, this is a reflexive partial order. Furthermore,

    \[0x = 0, x1 = x \rightarrow 0 \leq x, x \leq 1\]

    So we have greatest and least elements. We also have least upper bounds and
    greatest upper bounds:

    If $c \leq x, c \leq y$, then $c \leq xy \leq x, y$: By definition,
    $cx = c, cy = c$. Then,

    \[ cxy = cy = c \rightarrow c \leq xy\]
    \[ x(xy) = xy \rightarrow xy \leq x\]
    \[ y(xy) = (xy)y = xy \rightarrow xy \leq y\]


    If $x \leq c, y \leq c$, $x, y \leq x + xy + y \leq c$: By definition,
    $xc = x, yc = y$. Then,

    \[(x + y + xy)c = xc + yc + xyc = x + y + xy \rightarrow x + y + xy \leq c\]
    \[(x + y + xy)x = x + yx + yx = x \rightarrow x \leq x + y + xy\]
    \[(x + y + xy)y = xy + y + xy = y \rightarrow y \leq x + y + xy\]

    Denoting the operations $x \frown y = xy, x \smile y = x + y + xy$, these
    operations a

    This is object is called a boolean algebra. We will now proceed to define
    them by starting from a lattice and defining a ring structure. One advantage
    of that approach is that in order to verify a map between boolean algebras
    is a homomorphism, it will suffice to check that it respects the order
    relation ($x \leq y$ iff $f(x) \leq f(y)$). We will also need the definition
    based on rings to discuss ideals later.

    \section{}

    A partial order with least upper bounds ('joins') and greatest lower bounds
    ('meets') is called a lattice; a lattice with a greatest and least element
    is called a bounded latice. A lattice where $\smile$ and $\frown$ distribute
    over each other is called a a distributive lattice, and a lattive where, for
    every $x$ there's a $x'$ with $x \smile x' = 1$, $x \frown x' = 0$ is called
    a complemented lattice. Such a complement is unique -- let $x'$, $x''$ be
    complements of $x$. Then,

    \[x' \smile x'' = (x' \smile  x'') \frown 1\]
    \[= (x' \smile  x'') \frown (x \smile x'')\]
    \[= (x' \frown x) \smile x''\]
    \[= 0 \smile x'' = x''\]

    So, $x' \leq x''$, by applying the definition in reverse we get $x'' \leq
    x'$ so $x' \ x''$. In particular, this shows that the complement of the
    complement of $x$ is $x$.

    We can also use this to prove de Morgan's laws:

    \[(x' \smile y') \frown (x \frown y) = (x' \frown (x \frown y)) \smile (y'
    \frown (x \frown y))\]
    \[ = (0 \frown y) \smile (0 \frown x) = 0\]

    \[(x' \smile y') \smile (x \frown y) = ((x' \smile y') \smile x) \frown (x'
    \smile y') \smile y)\]
    \[ = (1 \smile y') \frown (1 \smile x') = 1\]

    So, $x' \smile y'$ is the complement of $x \frown y$:
    \[(x' \smile y') = (x \frown y)'\]
    Replacing $x$ and $y'$ by their respective complements, and taking the
    complement of both sides,
    \[(x'' \smile y'')' = (x' \frown y')''\]
    \[(x \smile y)' = x' \frown y'\]

    To make this a ring, the multiplication is defined as $\frown$, and the
    addition is defined as:

    \[ x + y = (x \frown y') \smile (x' \frown y)\]

    I'll omit the details, but this does form a ring, with $x^2 = x \frown x =
    x$, so a boolean algebra.

    Our previous definition of $\smile$ is compatible:


    \[A = \{X: X \subseteq \mathbb{N}: X \text{ or } \mathbb{N} - X \text{ is
    finite} \} \]

    \section{}

    A homomorphism of boolean algebras is a homomorphism of rings that happen to
    be boolean albegras -- maps $h$ so that

    \[h(x+y) = h(x) + h(y)\]
    \[h(xy) = h(x)h(y)\]
    \[h(1) = 1\]

    The condition $h(0)= 0$ can be derived from:

    \[h(0) = h(0+0) = h(0)+h(0) = 0\]

    From this definition, we can easily prove

    \[h(x \frown y) = h(xy) = h(x)h(y) = h(x) \frown h(y)\]
    \[h(x') = h(x+1) = h(x) + h(1) = h(x) + 1 = h(x)'\]
    \[h(x \smile y) = h(x + y + xy) = h(x) + h(y) + h(x)h(y) = h(x) \smile
    h(y)\]
    \[x \leq y \rightarrow xy = x \rightarrow h(xy) = h(x) \rightarrow h(x) \leq
    h(y)\]

    It also suffices to specify

    \[ h(x \frown y) = h(x) \frown h(y) \]
    \[h(x') = h(x)'\]

    Since
    \[x + y = (x \frown y') \smile (x' \frown y) = (x \frown y')' \frown (x'
    \frown y)'\]

    An isomorphism of Boolean algebras is a bijective homomorphism. An inverse
    of an isomorphism is also a homomorhpism:




    

\end{document}
 

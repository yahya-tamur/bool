\documentclass{article}

\usepackage{xurl}
\usepackage[utf8]{inputenc}
\usepackage{amsmath}
\usepackage{amssymb}
\usepackage{amsfonts}
\usepackage{enumitem}
\usepackage{amsthm}
\newtheorem*{remark}{Remark}
\newtheorem*{proposition}{Proposition}
\newtheorem*{lemma}{Lemma}
\newtheorem*{sublemma}{Sublemma}
\newtheorem*{definition}{Definition}
\DeclareMathOperator{\Hom}{Hom}

\renewcommand{\thesubsection}{\thesection.\alph{subsection}}

\title{Boolean Algebras}
\author{Yahya Tamur}
\date{Winter 2023}

\begin{document}
  \maketitle


  \begin{abstract}

    Boolean algebras are initially defined as rings where $x^2 = x$ or partial
    orders with certain properties. We then define the notions of filters and
    ideals, and show the bijection between maximal ideals, ultrafilters, and
    homomorphisms into $\{0,1\}$ from a Boolean algebra. These form the elements
    of the so called Stone space of a Boolean algebra. We then introduce the
    notion of Boolean topological spaces, and classify Boolean algebras and
    Boolean topological spaces in terms of each other. We conclude by showing a
    bijection between homomorphisms of Boolean algebras and continuous functions
    between their Boolean topological spaces, forming an equivalence of
    categories. This result is known as Stone's representation theorem, and
    while not explored here, can be extended to a more general result known as
    Stone duality.

  \end{abstract}

    \section{Uses of the Axiom of Choice}

      In this section, we'll discuss one proof of Zorn's Lemma assuming the
      axiom of choice, and then one consequence, Tychonoff's Theorem, and
      several basic facts about ultrafilters.

      We'll later use several important results from this section -- including
      the Ultrafilter Lemma, Tychonoff's theorem, and even the fact that in a
      compact Hausdorff space, every ultrafilter converges to a unique point.

      The outline of the proof of Zorn's Lemma comes from \cite{zorn}, and the
      outline of Tychonoff's Theorem comes from \cite{tych}.

      \subsection{Zorn's Lemma}

        The lemma states that if every totally ordered subset of a partial order
        $A$ has an upper bound, then the set must have a maximal element.

        Let $\leq$ be a partial order of the set $X$. Given a totally ordered
        subset $C$, define

        \[P(C,x) = \{y \in C: y < x\}\]

        Suppose every totally ordered subset of $X$ has an upper bound. Assume
        that $X$ must not have a maximal element. We'll show that this leads to
        a contradiction.

        Given any totally ordered subset $C$, we can find an upper bound $u$,
        and then, by assumption, an $x$ with $x>u$, so a strict upper bound.
        Using the axiom of choice, assign every totally ordered subset $C$, a
        strict upper bound $f(C)$.

        Call a subset $A$ of $X$ conforming if $\leq$ defines a well ordering on
        $A$, and for each $x \in A$, $x = f(P(A,x))$.

        \begin{lemma}
        If $A$ and $B$ are conforming subsets, $A \neq B$, then $A$ must
        be $P(B,x)$ for some $x$ or vice versa.
        \end{lemma}

        \begin{proof}
        If $A \neq B$, either $A \setminus B$ or $B \setminus A$ is
        nonempty, without loss of generality, assume $A \setminus B \neq
        \emptyset$.

        Then, since $A$ is well-ordered, $A \setminus B$ has a least element
        $x$. If $k \in A, k < x$, by the minimality of $x$, $k$ must be in $B$.
        Then,

        \[P(A,x) \subseteq B\]

        We'll prove by contradiction that this is an equality. Assume that $B
        \setminus P(A,x)$ is nonempty. Then, it has a least element $y$.

        \begin{sublemma}
          For all $u \in P(B,y)$, for all $v \in A$, if $v < u$, then
          $v \in P(B,y)$.
        \end{sublemma}

        \begin{proof}
        If $v$ is not in $B$, $v \in A \setminus B$, and $x$ is the least
        element of that, so $x \leq v$. Then,

        \[u > v \geq x\]

        Since $u$ is in $B$, and $u$ isn't less than $x$,

        \[u \notin P(A,x) \text{ , and } u \in B \setminus P(A,x)\]

        However, $u < y$ because $u \in P(B,y)$, and this contradicts the
        minimality of $y$ in $B \setminus P(A,x)$.

        We can conclude that for any $v \in P(A,u)$, $v \in B$. Since $v < u <
        y$, $v \in P(B,y)$.
        \end{proof}

        Now, let $z$ be the least element of $A \setminus P(B,y)$. Note that

        \[A \setminus P(B,y) \supseteq A \setminus B\]

        So, first of all, the latter is nonempty, so $z$ exists. Since they're
        the minimal elements of the respective sets, $z \leq x$. Then, $z \in
        P(A,x) \subseteq B$.

        \begin{sublemma}
        P(A,z) = P(B,y)
        \end{sublemma}

        \begin{proof}
        If $w \in P(A,z)$, $w \in A$, $w < z$, so $w \notin A \setminus P(B,y)$
        by the minimality of $z$ in that set. Then, $w \in P(B,y)$.

        If $w \in P(B,y)$, since $z \in B$, $w$ and $z$ must be comparable,
        since $B$ is well-ordered. If $z < w$, $z \in P(B,y)$ by the previous
        sublemma.

        This is impossible, since $z \in A \setminus P(B,y)$. $z = w$ also
        implies $z \in P(B,y)$. Then, we conclude that $w < z$.

        Now, $y$ is minimal in $B \setminus P(A,x)$.

        \[w \in B \text{ , } w < y \text{ , } w<z \leq x\]

        So, $w$ must be in $A$ to not contradict the minimality of $y$. Then,
        $w \in P(A,z)$.

        \end{proof}

        Since $A$ and $B$ are conforming,

        \[z = f(P(A,z)) = f(P(B,y)) = y\]

        Since $x \in A \setminus B$, $y \in B$ implies $z \neq x$. Then, $z <
        x$, $z \in A$, so $z \in P(A,x)$. But, $y$ should be in $B \setminus
        P(A,x)$, which is the contradiction.
        \end{proof}

        Now, if $A$ is a conforming subset of $X$, $a \in A$, and $y < a$, $y
        \notin A$, $y$ cannot be in any conforming subset $B$ -- then, since
        either $A$ is a subset of $B$ or vice versa, and $y \in B \setminus A$,
        $A$ must be $P(B,x)$ for some $x$. Then, $y < a < x$, which means $y$
        should be in $A$.

        Let $U$ be the union of all conforming subsets of $X$. Then, any subset
        of $U$ must be contained in a single conforming subset -- assume by
        contradiction there is a subset $S$ not contained in a single conforming
        set. Then, it must have two elements $a, b$ with $a \in A - B$, $b \in B
        - A$ for conforming subsets $A, B$. This is a contradiction, since we've
          proved that given any two conforming subsets, one must be a subset of
          the other. So, any subset of $U$ must be well-ordered.

        Also, for all $x \in U$, $x \in A$ for some conforming subset $A$, so $x
        = f(P(A,x))$. But then, if $u \in U$, $u < x$, $u \in A$ since otherwise
        $u$ can not be in any other conforming subset, so it can't be in $U$.
        So, \[P(A,x) = \{a \in A: a < x\} = \{u \in U: u < x\} = P(U,x)\] \[x =
        f(P(A,x)) = f(P(U,x))\]

        So, $U$ is a conforming subset.

        Let $x = f(U)$. Since $x$ is greater than any element of $U$, $U \cup
        \{x\}$ is still well-ordered. For any $y \in U$,
        \[P(U,y) = P(U \cup\{x\},y)\]

        Since $u \in U$ is in either set iff $u < y$, and $x$ is in neither.

        Then,

        \[y = f(P(U,y)) = f(P(U \cup \{x\}, y))\]
        \[x = f(U) = f(P(U \cup \{x\}, x))\]


        So $\{x\} \cup U$ is a conforming subset, which contradicts the
        assumption that $U$ was the union of all the conforming subsets. Then,
        $X$ must have a maximal element.

      \subsection{Tychonoff's Theorem}

        Define a filter $F$ as a set of subsets of some set $S$, with:

        \[ \emptyset \notin F, S \in F\]
        \[ x, y \in F \rightarrow x \cap y \in F\]
        \[ x \in F, y \supseteq x \rightarrow y \in F\]

        To look even more like the definition of the ideal in ring theory, we
        could rephrase the third condition as $x \in F, y \in \mathcal{P}(S) \to
        x \cup y \in F$.

        An ultrafilter is a filter, not strictly included in any other filter.
        In particular, this implies that for any subset $A$ of $S$, if $A$ is
        not in some ultrafilter $F$, then, $A$ must have a zero intersection
        with some element in $F$ -- otherwise, the set $F \cup \{A\}$ completed
        under the operations of finite intersection and arbitrary union still
        doesn't include the empty set, and is therefore a filter. The converse
        is clearly true as well -- every element in the set has a nonempty
        intersection with every element -- so this completely characterizes the
        elements of an ultrafilter.

        \begin{proposition}
        (the Ultrafilter Lemma) Every filter is included in some ultrafilter.
        \end{proposition}

        \begin{proof}
        Let $F$ be some filter. Let

        \[E = \{\text{ filter } G \text{ of } S: F \subseteq G\}\]

        Under the partial order of inclusion. Let $C$ be some totally ordered
        subset of $E$. If $C$ is empty, $F$ is an upper bound of $C$ included in
        $E$. If it's nonempty, consider

        \[F_0 = \bigcup_{G \in C} G\]

        $\emptyset \notin F_0$, $S \in F_0$ since it's a nonempty union of
        filters. If $x, y \in F_0$, $x \in G_1, y \in G_2$ for some $G_1, G_2
        \in C$. Since $C$ is totally ordered, one is included in another, so
        without loss of generality, assume $G_1 \subseteq G_2$. Then, $x,y \in
        G_2$, and \[x \cap y \in G_2 \subseteq F_0\]

        If $x \in F_0, y \supseteq x$, then $x \in G_1$ for some $G_1 \in C$,
        and then $y \in G_1 \subseteq F_0$.

        So, $F_0$ is a filter. Since it's a nonempty union, it contains $F$.
        Then, it's an upper bound of $C$ in $E$.

        By Zorn's Lemma, we conclude that there must be some maximal element of
        $E$, so a filter $F_1$ including $F$ not contained in any other filter.
        This is exactly what an ultrafilter is, so the proof is complete.
        \end{proof}

        Everything up to now works when $A$ is an arbitrary set, but from now
        on, we'll assume $A$ is a topological space.

        Compactness is usually stated:

        Every open cover has a finite subcover.

        By taking the complements, we get:

        Every infinite set of closed sets with an empty intersection has a
        finite subset which still has an empty intersection.

        By taking the contrapositive, we get:

        Let $C$ be any set of closed sets. If every finite subset of $C$ has
        a nonempty intersection (A closed cover satisfying this is said to have
        the finite interseciton property), then $C$ has a nonempty intersection
        as well.

        An ultrafilter is said to converge to a point $x$ if every open
        neighborhood of $x$ is an element of the ultrafilter.

        We'll use this definition to prove the following:

        \begin{proposition}
        A set is compact if and only if every ultrafilter of the
        set converges to a point.
        \end{proposition}

        \begin{proof}

          Assume every ultrafilter converges to a point, and let $C$ be some
          closed cover with the finite intersection property.

          Then, let

          \[F = \left\{A \subseteq S: A \supseteq \bigcap_{i=1}^n C_i, \forall
          i, C_i \in C, n
          \in \mathbb{N}\right\}\]

          Since $C$'s intersections are nonempty, this is a filter, and it's
          contained in some ultrafilter. That ultrafilter converges to some
          point $x$.

          Then, let $B$ be some closed set in $C$. Since every open neighborhood
          of $x$ is in the same ultrafilter as $B$, the intersection of any open
          neighborhood of $x$ and $B$ is nonempty.

          So, $x$ can not be in $B^c$ since that's an open set disjoint with
          $B$ -- then, $x$ must be in $B$. We conclude that $x$ is in every
          element of $C$, and that the intersection of all the elements of $C$
          is nonempty.

          Since every closed cover with the finite intersection property has a
          nonempty intersection, the set $S$ is compact.

          Now, assume $C$ is compact, and let $U$ be some ultrafilter. Then,

          \[\{\overline{A}: A \in U\}\]

          Is a collection of closed sets, covers $S$ since $S$ itself is in $U$,
          and has the finite intersection property -- every finite subset
          $\{\overline{A_i}\}_{i=1}^n$ has a nonempty intersection, since the
          corresponding elements $\{A_i\}_{i=1}^n$ have a nonempty intersection,
          which happens because $U$ is a filter.

          By compactness, the intersection of every element in this set is
          nonempty. Let $x$ be some element of this intersection, and $O$ be any
          open set containing it.

          We'll show that any set $B$ disjoint with $O$ can not be in the
          ultrafilter:

          If a set $B$ is disjoint with $O$, $O^c$ is a closed set containing it,
          and therefore $O^c \supseteq \overline{B}$. Then, $\overline{B}$
          doesn't have $x$, and $B$ can not be in the ultrafilter.

          As we've discussed, every set with a nonempty intersection with every
          element of the ultrafilter must be in the ultrafilter by its
          maximality. Then, $O$ must be in the ultrafilter.

          So, $U$ contains all the open sets containing
          $x$, and therefore converges at $x$.
          \end{proof}

          \begin{proposition}
          If and only if $S$ is Hausdorff, every ultrafilter can converge at at
          most one point.
          \end{proposition}

          \begin{proof}
          Assume $S$ is Hausdorff, and an ultrafilter $U$ converges at distinct
          $x$ and $y$. Then, $x$ and $y$ have disjoint open neighborhoods since
          $S$ is Hausdorff, $U$ must contain both of them, but cannot contain
          disjoint sets. This is a contradiction.

          Now, assume $S$ is not Hausdorff. There must be $x$, $y$ where any
          open neighborhood of $x$ intersects with any open neighborhood of
          $y$. Consider

          \[U = \{A \text{ open }: x \in A \text{ or } y \in A\}\]

          This has the finite intersection property, and includes $S$ but not
          $\emptyset$. So, the set of sets containing an element of $U$ is a
          filter, and can be extended to an ultrafilter. This ultrafilter then
          converges to both $x$ and $y$.
          \end{proof}

          Given any function $f: X \to Y$, and an ultrafilter $U$ on $X$, we can
          define the following collection of sets:

          \[f_*U = \{A \subseteq Y: f^{-1}(A) \in U\}\]

          \begin{lemma}
          $f_*U$ is an ultrafilter.
          \end{lemma}

          \begin{proof}
          It's pretty easy to see that the inverse image map preserves
          intersections and inclusions.

          $f_*U$ has $Y$ ($f^{-1}(Y) = X \in U$), but not $\emptyset$
          ($f^{-1}(\emptyset) = \emptyset \notin U$).

          If $A$ and $B$ are sets in $f_*U$, $A = f^{-1}(V)$, $B = f^{-1}(W)$ for
          $V, W \in U$. Then, $A \cap B = f^{-1}(V \cap W)$ is in $U$, so is
          nonempty. By induction, $f_*U$ has the finite intersection property.

          If $A \in f_*U$, $B \supseteq A$, $f^{-1}(B) \supseteq f^{-1}(A) \in U$,
          so $f^{-1}(B)$ is in $U$ as well, and then $B$ is in $f_*U$.

          Finally, we need to prove that this is an ultrafilter. We'll use the
          following lemma:

          \begin{sublemma}
          For any set $A$ and ultrafilter $U$, $U$ contains $A$ or $A^c$. If any
          filter $U$ contains every set or its complement (but not both), it's
          an ultrafilter.
          \end{sublemma}

          \begin{proof}
          If $U$ doesn't contain $A$, by the maximality of $U$, $U$ must have
          some element $B$ with an empty intersection with $A$. This $B$ is then
          contained in $A^c$.

          Similarly, if $U$ doesn't contain $A^c$, it must contain some element
          $C$ disjoint from $A^c$. Then, $C$ must be in $A$.

          But then, $B$ and $C$ have empty intersection, so $U$ cannot contain
          both of them.

          So, $U$ must contain either $A$ or $A^c$. Note that it can not contain
          both either -- it must contain exactly one of them.

          Now, let $U$ be some filter that contains exactly one of $A$
          or $A^c$ for every $A$. It's maximal, since if there's an $A$ it
          doesn't contain, it contains $A^c$, and can't contain $A$ without
          having empty finite intersections.
          \end{proof}

          Let $A$ be any subset of $Y$. $U$ must then contain $f^{-1}(A)$ xor
          $f^{-1}(A)^c = f^{-1}(A^c)$, so $f_*U$ must contain $A$ xor $A^c$. Then,
          $f_*U$ is an ultrafilter.
          \end{proof}

          We are finally ready to prove Tychonoff's theorem.

          \begin{proposition}
            Let $x = \{x_i\}_{i \in I}$ be a point in the product space $X
            =\Pi_{i \in I} X_i$. An ultrafilter $U$ on $X$ converges to $x$ if
            and only if ${\pi_i}_*U$ converges to $x_i$ for all $i$.
          \end{proposition}

          \begin{proof}
          
            If $U$ converges to $x$, for any open neighborhood $O_i$ of $x_i$
            in $X_i$, the set $\pi_i^{-1}O_i$ is a basic open set in the
            product topology containining $x$:

            \[\pi_i^{-1}O_i = \prod_{j \in I} \text { if } i = j \text{ then } O_i
            \text{ else } X_i\]

            And is therefore in $U$. Then, $O_i$ is in ${\pi_i}_*U$. So,
            ${\pi_i}_*U$ converges to $x_i$.

            And, if ${\pi_i}_*U$ converge to $x_i$, all neighborhoods of $x_i$,
            $O_i$ are in ${\pi_i}_*$, so $\pi_i^{-1}(O_i)$ are in $U$. These
            have the form above, and all their finite intersections, which
            must also be in $U$, are the elements of the basis of $X$
            containing $x$. Then, $U$ contains every open set containing $x$,
            and therefore converges to $x$.

          \end{proof}

          It immediately follows that an ultrafilter converges to multiple
          points if and only if one of the ${\pi_i}_*U$ converges to multiple
          points. If all the $X_i$ are Hausdorff, none of them may converge at
          multiple points, so no ultrafilter of $X$ can converge at multiple
          points either, and $X$ must be Hausdorff.

          Similarly, if all the $X_i$ are compact, given any ultrafilter $U$,
          all the ${\pi_i}_*$ must converge at some point. We need to invoke the
          axiom of choice to find an $x$ such that $x_i$ is a convergence point
          of ${\pi_i}_*$. Then, $U$ converges to $x$. So, $X$ is compact. This
          result is known as Tychonoff's theorem.

        % mention compactness for logic?

  \section{Boolean Algebras}

    Most of the rest of the paper comes from chapter 2 of \cite{bool}. The last
    section comes from an exercise at the end.

    \subsection{Boolean Algebras as Rings}

      First, we'll define Boolean algebras, first as a ring where $x^2=x$, then as
      a partial order with certain properties. We'll show that both define the
      same objects. Both approaches will be useful later on.

      Let $B$ be a ring where $x = x^2$ for all $x$. Some immediate consequences
      are:

      Every element is its own additive inverse:

      \[ x + x = (x + x)^2 = x^2 + x^2 + x^2 + x^2 = x + x + x + x\] \[0 = x + x\]

      Multiplication is commutative:

      \[ x + y = (x + y)^2 = x^2 + xy + yx + y^2 = x + xy + yx + y\] \[0 = xy +
      yx\] \[xy = -yx = yx\]

      Where the last identity follows from our previous result.

      Now, define the following binary relation: $x \leq y$ iff $xy = x$. We get:

      \[xx = x \text{, so } x \leq x\]

      If $x \leq y$, $y \leq z$, \[xy = x, yz = y\] \[xz = x(yz) = (xy)z = yz = x
      \text{, so } x \leq z\]

      And, if $x \leq y$, $y \leq x$,

      \[x = xy = yx = y\]

      So, this is a reflexive partial order. Furthermore,

      \[0x = 0, x1 = x\]

      So, $0 \leq x$, $x \leq 1$.

      We also have least upper bounds and greatest upper bounds:

      We'll claim that for every $c \leq x,y$, $c \leq xy \leq x,y$.

      If $c \leq x, c \leq y$, then $cx = x$, $cy = y$. Then,

      \[ cxy = cy = c\]

      So $c \leq xy$.

      \[ x(xy) = xy\]

      So $xy \leq x$.

      \[ y(xy) = (xy)y = xy\]

      And then $xy \leq y$.

      Similarly, we'll claim that if $x,y \leq c$, $x,y \leq x + y + xy \leq c$.
      If $x \leq c, y \leq c$,

      \[(x + y + xy)c = xc + yc + xyc = x + y + xy\]

      So $x + y + xy \leq c$.

      \[(x + y + xy)x = x + yx + yx = x\]

      So $x \leq x + y + xy$.

      \[(x + y + xy)y = xy + y + xy = y\]

      So, finally, $y \leq x + y + xy$.

      Denoting the operations $x \frown y = xy, x \smile y = x + y + xy$, these
      operations form a lattice -- a partial order with least upper bounds and
      greatest lower bounds.

      This object is called a Boolean algebra. We will now proceed to define
      Boolean algebras by starting from a lattice and defining a ring structure.
      One advantage of that approach is that in order to verify a map between
      boolean algebras is a homomorphism, it will suffice to check that it
      respects the order relation ($x \leq y$ iff $f(x) \leq f(y)$). We will also
      need the definition based on rings to discuss ideals later.

      One obvious example of a Boolean algebra is all subsets of some set $X$,
      with $\smile$ corresponding to $\cup$, $\frown$ corresponding to $\cap$,
      $0$ corresponding to $\emptyset$, and $1$ corresponding to the whole set
      $X$.

      $+$ corresponds to the symmetric difference, sometimes denoted
      $\triangle$, and $1 + x$ corresponds to the complement of $x$.

      We can also only select certain subsets from a set $X$, provided that
      those subsets are closed under all of these operations. As part of the
      classification of we'll do later, we'll see that every Boolean algebra can
      be expressed this way.

    \subsection{Boolean Algebras as Lattices}

      Let $A$ be a lattice with a minimum $0$ and maximum $1$, where, for every
      $x$ there's a $x'$ with

      \[x \smile x' = 1\] \[x \frown x' = 0\]

      Such a complement is unique -- let $x'$, $x''$ be complements of $x$. Then,

      \[x' \smile x'' = (x' \smile  x'') \frown 1\] \[= (x' \smile  x'') \frown (x
      \smile x'')\] \[= (x' \frown x) \smile x''\] \[= 0 \smile x'' = x''\]

      So, $x' \leq x''$, by applying the definition in reverse we get $x'' \leq
      x'$ so $x' = x''$. In particular, this shows that the complement of the
      complement of $x$ is $x$.

      We can also use this to prove de Morgan's laws:

      \[(x' \smile y') \frown (x \frown y) = (x' \frown (x \frown y)) \smile (y'
      \frown (x \frown y))\] \[ = (0 \frown y) \smile (0 \frown x) = 0\]
      \[(x' \smile y') \smile (x \frown y) = ((x' \smile y') \smile x) \frown (x'
      \smile y') \smile y)\] \[ = (1 \smile y') \frown (1 \smile x') = 1\]

      So, $x' \smile y'$ is the complement of $x \frown y$: \[(x' \smile y') = (x
      \frown y)'\] Replacing $x$ and $y'$ by their respective complements, and
      taking the complement of both sides, \[(x'' \smile y'')' = (x' \frown
      y')''\] \[(x \smile y)' = x' \frown y'\]

      To make this a ring, the multiplication is defined as $\frown$, and the
      addition is defined as:

      \[ x + y = (x \frown y') \smile (x' \frown y)\]

      Clearly, $x^2 = x$, $x + x = 0$. I'll omit the proof for associativity,
      distributivity, etc. but this does form a ring. We should also check that
      the ring that it defines forms the same Boolean algebra as the lattice.

      To do this, we'll look at the $\leq$ operation. As a ring, $x \leq' y$ if
      and only if $x*y = x$, so $x \frown y = x$. But, the greatest lower bound of
      $x$ and $y$ is $x$ if and only if $x \leq y$. Since a lattice is just a
      partial order with certain properties, this means the Boolean algebra
      defined with these ring operations is the exact Boolean algebra defined
      using the lattice.

      Also, the complement of an element $x$ in the lattice is expressed as
      $1+x$ in the ring. Immediately,

      \[x \frown (1+x) = (1+x)x = x + x^2 = x + x = 0\]
      \[x \smile (1+x) = x + 1 + x + x(1+x) = 1 + x + x + 0 = 1\]
      \[1 + (1 + x) = 1 + 1 + x = x\]

    \section{Homomorphisms and Isomorphisms}

    A homomorphism of Boolean algebras are maps that preserve every property we
    defined on a Boolean algebra. For a map $h$ to be a homomorphism, it
    suffices to show that:

    \[h(x+y) = h(x) + h(y)\]
    \[h(xy) = h(x)h(y)\]
    \[h(1) = 1\]

    The condition $h(0) = 0$ can be derived from:

    \[h(0) = h(0+0) = h(0)+h(0) = 0\]

    From this definition, we can prove

    \[h(x \frown y) = h(xy) = h(x)h(y) = h(x) \frown h(y)\]
    \[h(x') = h(x+1) = h(x) + h(1) = h(x) + 1 = h(x)'\]
    \[h(x \smile y) = h(x + y + xy) = h(x) + h(y) + h(x)h(y) = h(x) \smile
    h(y)\]
    \[x \leq y \rightarrow xy = x \rightarrow h(xy) = h(x) \rightarrow h(x) \leq
    h(y)\]

    It also suffices to specify

    \[ h(x \frown y) = h(x) \frown h(y) \]
    \[h(x') = h(x)'\]

    Since

    \[x + y = (x \frown y') \smile (x' \frown y) = (x \frown y')' \frown (x'
    \frown y)'\]
    \[h(1) = h(0') = h(0)' = 0' = 1\]

    An isomorphism of Boolean algebras is a bijective homomorphism. For a
    surjective $h$, the following condition is necessary and sufficient to
    define an isomorphism:

    \[x \leq y \text{ if and only if } h(x) \leq h(y) \]

    In an isomorphism, this holds, since

    \[xy = x \text{ if and only if } h(x)h(y) = h(x)\]

    Conversely, if this holds, $h$ is injective -- if $h(x) = h(y)$,

    \[h(x) \leq h(y) \rightarrow x \leq y\]
    \[h(y) \leq h(x) \rightarrow y \leq x\]
    \[x = y\]

    We can prove that $x \frown y = h(x \frown y)$ using the greatest lower
    bound property:

    Let $z$ be the number such that $h(z) = h(x) \frown h(y)$. Then,

    \[h(z) \leq h(x) \rightarrow z \leq x\] \[h(z) \leq h(y) \rightarrow z \leq
    y\]

    So, $z \leq x \frown y$.

    \[x \frown y \leq x \rightarrow h(x \frown y) \leq h(x)\]
    \[x \frown y \leq y \rightarrow h(x \frown y) \leq h(y)\]

    \[h(x \frown y) \leq h(x) \frown h(y) = h(y)\]
    \[x \frown y \leq z\]

    We can prove that $h(x \smile y) = h(x) \smile h(y)$ using a similar
    argument.

    $h(1)$ is greater than or equal to every element, so $h(1) = 1$, and
    similarly $h(0)$ is $0$.

    Finally,

    \[h(x') \smile h(x) = h(x' \smile x) = h(1) = 1\]
    \[h(x') \frown h(x) = h(x' \frown x) = h(0) = 0\]

    So $h(x') = h(x)'$

    Then, $h$ is a bijective homomorphism as claimed.

    This also shows that the inverse of an isomorphism is still an isomorphism,
    since the condition $x \leq y \text{ iff } h(x) \leq h(y)$ implies the same
    thing for the inverse.

    \section{Ideals}


      \begin{proposition}

        For a subset $I$ of a Boolean algebra $A$ is a proper ring ideal if and
        only if it's a partial order ideal -- a subset with

        \[0 \in I, 1 \notin I\]
        \[\forall x, y \in I, x \smile y \in I\]
        \[\forall x \in I, y \in A, y \leq x \implies y \in I\]

      \end{proposition}

      \begin{proof}

      Let $I$ be a proper ring ideal. $0$ is in $I$, since it's an additive
      subgroup. $1$ isn't, since then, for every $a \in A$, $a*1 = a$ is in $I$,
      then $I$ would have to be the entire ring.

      If $x$ and $y$ are in $I$, by definition, $xy$ is too. Then, since $I$ is
      an additive subgroup, $x \smile y = x + y + xy$ is as well.

      Finally, if $x$ is in $I$, $xy \in I$. $y \leq x$, by definition means $xy
      = y \in I$.

      Conversely, I be a partial order ideal.

      If $x, y \in I$, $x \smile y \in I$.

      \[(x+y)(x \smile y) = (x+y)((x+y)+xy) = x + y + (x+y)xy = x + y + xy + xy
      = x + y\]

      So, $x+y
      \leq x \smile y$, and $x+y \in I$ by the third condition. Since $-x = x$
      in a boolean algebra, we conclude that $I$ is an additive subgroup.

      For $x \in I$, $xy = x \frown y \leq x$, so $xy \in I$. So, $I$ is an
      ideal. It's proper, since it doesn't include the element $1$.
      \end{proof}

      \begin{proposition}
        The following are equivalent:

        \begin{itemize}
          \item[(1)] $I$ is a maximal ideal
          \item[(2)] $A/I$ is isomorphic to $\{0,1\}$
          \item[(3)] $I$ is the kernel of a homomorphism$ A \rightarrow \{0,1\}$
          \item[(4)] For all $x$, $x \in I$ or  $x+1 \in I$
          \item[(5)] For all $x,y$ if $xy \in I$, then $x \in I$ or $y \in I$
        \end{itemize}
      \end{proposition}

      \begin{itemize}[align = left]
      \item[$(1) \rightarrow (2)$]
      In fact, in any commutative ring, we can prove that $A/I$ is a field if
      and only if $I$ is maximal:

      If $I$ isn't maximal, let $I \subsetneq J$, and $a \in J - I$. $a$ isn't
      in $I$, so $[a]$ isn't the zero element in $A/I$. If it was invertible,
      there would be a $b$ with

      \[[a][b] = 1\]
      \[[ab - 1] = 0\]
      \[ab - 1 \in I\]

      $a \in I$, so $ab \in I$. Then, \[ab - (ab - 1) = 1 \in I\] This is a
      contradiction. We conclude that $A/I$ has an non-invertible elements, so
      it can't be a field.

      Conversely, assume $I$ is maximal. Let $[a] \neq 0 \in A/I$. Form \[K =
      \{ay + z: y \in A, z \in I\} \] Clearly, $0 = a*0 + 0 \in K$, if $ay_1 +
      z_1$, $ay_2 + z_2 \in K$, since the ring is commutative, \[ay_1 + z_1
      -(ay_2 + z_2) = a(y_1-y_2) + (z_1 - z_2) \in K\] And, if $ay + z \in K$,
      \[(ay + z)t = a(yt) + (zt) \in K\] So, $K$ is an ideal. It includes any
      element $a*0 + y$ of $I$, and it includes $a*1 + 0 = a$, which isn't in
      $I$. Since $I$ is maximal, $K$ must be the whole ring, and then, there
      must be $y, z$ with \[ay + z = 1\] Then, \[ay + z= [a][y] + [z] = [a][y] =
      1\] Note that $z \in I$. Then, any nonzero element of $A/I$ does have an
      inverse, and this proof is complete.

      Finally, note that $\{0,1\}$ is the only Boolean algebra which is a field:
      For every $x$, \[(x+1)x = x + x = 0\] Which, if this is a field (or even a
      domain), implies that either $x=0$ or $x=1$.

      \item[$(2) \rightarrow (3)$] $I$ is the kernel of the quotient map $A
      \rightarrow A/I = \{0,1\}$.

      \item[$(3) \rightarrow (4)$] Let $h$ be the homomorphism $A \rightarrow
      \{0,1\}$ with kernel $I$. Then, for all $x$, $h(x) = 0$ or $h(x) = 1$. In
      the latter case, $h(1+x) = h(1) + h(x) = 1 + 1 = 0$. So, $x$ or $1+x$ is
      in the kernel, which is $I$.

      \item[$(4) \rightarrow (5)$] Let $x, y \notin I$. Then, $1+x, 1+y \in I$.
      Then, \[(1+x) \smile (1+y) = 1 + (x \frown y) = 1 + xy \in I\] So, $xy \in
      I$ implies $x \in I$ or $y \in I$.

      \item[$(5) \rightarrow (1)$] Suppose $I$ isn't maximal. Then, let $I
      \subsetneq J$ be an ideal, $a \in J - I$. $a \in J$, so $1+a \notin J$,
      because then \[a \smile (1+a) = 1 \in J\] But, $(1+a) \in I \subseteq J$,
      which is a contradiction.

      \end{itemize}


      The fact that maximal ideals are the kernels of homomorphisms $A
      \rightarrow\{0,1\}$ will be important later on -- the elements of the
      Stone space of a boolean algebra $A$ are homomorphisms $A \to \{0,1\}$.

    \section{Filters}

      A filter is a subset $F$ of a Boolean algebra $A$ such that:

      \[0 \notin F, 1 \in F\]
      \[x, y \in F \rightarrow x \frown y \in F\]
      \[x \in F, y \geq x \rightarrow y \in F\]

      When defined on the Boolean algebra of all subsets of some set, this is
      exactly the notion of the filter we've defined before.

      \begin{proposition}

      In a Boolean algebra, a subset $F$ is a filter if and only if $\{x \in
      A: 1+x \in F\}$ is an ideal.

      \end{proposition}

      \begin{proof}

      Let \[I = \{x \in A: 1+x \in F\}\]

      Assume the $F$ is a filter.

      The first condition implies $0 \in I, 1 \notin I$.

      The second condition implies for $x, y \in I$, $(1+x), (1+y) \in F$, then

      \[(1+x) \frown (1+y) \in F\]
      \[1 + (x \smile y) \in F\]
      \[x \smile y \in I\]

      Finally, notice that $x \leq y$ imples $1+y \leq 1+x$:

      \[xy = x\]
      \[(1+x)(1+y) = 1 + x + y + xy = 1 + x + y + x = 1 + y\]

      Then, if $x \in
      I$, $y \leq x$, $1+x \in F$, $1+y \geq 1+x$, so by the third condition,
      $1+y \in F$, and $y \in I$.  So, $I$ is an ideal.

      Conversely, if $I$ is an ideal, the proof is entirely the same:

      The first condition is given by $0 \in I, 1 \notin I$.

      If $x, y \in F$, $1+x, 1+y \in I$, then \[(1+x) \smile (1+y) \in I\] \[1 +
      (x \frown y) \in F\] \[x \frown y \in F\]

      And finally, if $x \in F$, $y \geq x$, \[1+x \in I, 1+y \leq 1+x
      \rightarrow 1+y \in I\] \[y \in F\]
      \end{proof}

      We can then apply what we know about maximal ideals to ultrafilters in
      Boolean algebras.

      \begin{proposition}
        The following are equivalent:

        \begin{itemize}
          \item[(1')] $F$ is an ultrafilter.
          \item[(3')] There's a homomorphism $g: A \rightarrow \{0,1\}$ with $F
          = h^{-1}(\{1\})$
          \item[(4')] For all x, $x \in F$ or $1+x \in F$
          \item[(5')] If $x \smile y \in F$, then $x \in F$ or $y \in F$
        \end{itemize}
      \end{proposition}

      \begin{proof}

        We have proven (4') for any ultrafilter in the first section.
        Regardless, we can prove all of these items by showing that each holds
        if and only if the corresponding identity in the previous section holds
          for the dual ideal $\{x \in A: 1+x \in F\}$.

      \begin{itemize}[align = left]

        \item[(1')] Saying there's no filter $F' \supsetneq F$ is equivalent to
        saying there's no ideal $I' \supsetneq I$, since a set is a filter if
        and only if the set of its complement's an ideal and vice versa.

        \item[(3')] For a boolean algebra homomorphism, $h(x+1) = h(x) + 1$, so
        a homomorphism sending $F$ to 1 is equivalent to a homomorphism sending
        $I$ to 0.

        \item[(4')] This is obvious by the definition of $I$ -- $x \in I$ if and
        only if $x+1 \in F$, etc.

        \item[(5')] Assume $(5')$ for $F$.

          For $x \frown y \in I$,

          \[1 + (x \frown y) = (1+x) \smile (1+y)\]
          
          is in $F$, so $1+x$ or $1+y$ is in $F$, and $x$ or $y$ is in $I$.

          Similarly, if (5) holds for $I$, let $x \smile y\in F$.

          \[I \ni 1 + (x \smile y) = (1 + x) \frown (1 + y)\]
          \[1 + x \in I \text{ or } 1 + y \in I\]

          So $x$ or $y$ is in $F$.
        \end{itemize}
      \end{proof}


      Continuing with the remark from last section, the property $3$ will be
      especially important, since the elements of the Stone space of the
      Boolean algebra will be homomorphisms $A \to \{0,1\}$. Property 3 says
      they also correspond bijectively with ultrafilters -- any ultrafilter is
      $h^{-1}(1)$ for such a homomorphism, and every such preimage of a
      homomorphism is an ultrafiler.

      \begin{remark}
      For any $x \neq 0 \in A$, we can form a filter including $x$:
      \[ F = \{y \in A: y \geq x\}\]

      It's easy to see that $0 \notin F$, $1 \in F$, $u, v \in F$ implies $u
      \frown v \in F$, and $u \in F, v \geq u$ implies $v \in F$, so it's indeed
      a filter. Then, using the Ultrafilter lemma, every element in $x$ is
      included in some ultrafilter.

      Equivalently, for any $x \in A$, we can form a homomorphism $A \rightarrow
      \{0,1\}$ with $h(x)=1$.
      \end{remark}

      Since filters and ideals biject a Boolean algebra, we could have also used
      another form of the axiom of choice from ring theory to prove this --
      Krull's theorem states that every ideal in a commutative ring is included
      in some maximal ideal, and can easily be proved using Zorn's Lemma.


    \section{The Stone space of a Boolean algebra}

        A topological space is said to be zero dimensional if it has a basis
        consisting of closed sets.

        We can equivalently say that the family of sets which are both open and
        closed ("clopen") form a basis for the topology. This implies
        the previous condition, and if the previous statement is true, the basis
        consisting of closed sets is a subset of the set of clopen sets, so the
        latter forms a basis as well.

        One example of a zero dimensional topological space is the set of
        rational numbers under the subspace topology from $\mathbb{R}$.

        A compact, Hausdorff zero dimensional topological space is called a
        Boolean topological space.

        Take the discrete topology on two points $\{0,1\}$, and form the product
        space $\{0,1\}^I$ for some indexing set $I$. A basic open set $\Omega$
        is either empty (in case when one of the sets in the product is the
        empty set), or has a finite number of indices which are either $\{0\}$
        or $\{1\}$.  With the elements written as functions $I \rightarrow
        \{0,1\}$, it's the set of functions whose values at a finite number of
        points is determined:

        \[\{f: I \rightarrow \{0,1\}: f(i_1) = \epsilon_1, ..., f(i_n) =
        \epsilon_n\}\]

        Where $n$ is some natural number and each $\epsilon_i$ is either $0$ or
        $1$. By De Morgan's Law, the complement of this set is

        \[\bigcup_{1 \leq j \leq n} \{f : f(i_j) = 1 - \epsilon_j\}\]

        This is a finite union of open sets in the basis, so it's open. Then,
        $\Omega$ is closed as well. Since every set in the basis is clopen, this
        is a zero dimensional topological set. $\{0,1\}$ is clearly compact and
        Hausdorff. As we've discussed in the first section, this implies that
        $\{0,1\}^I$ is compact and Hausdorff as well. So, it's a Boolean
        topological space.


      For a Boolean algebra $A$, the subset of $\{0,1\}^A$ which consists of
      homomorphisms $A \rightarrow \{0,1\}$ is denoted $S(A)$ and called the
      Stone space of $A$.

      Since $\{0,1\}^A$ is zero dimensional, we can find a basis by
      clopen sets. With $S(A)$ under the subspace topology, the same basis forms
      a basis of clopen sets, so $S(A)$ is zero dimensional as well.

      \begin{proposition}
        The sets in the basis of $S(A)$ are exactly the sets defined
        as \[\{h\in S(A): h(a) = 1\}\] for some $a \in A$. This element $a$ is
        unique.
      \end{proposition}

      \begin{proof}
      Note that the basis of $S(A)$ under the subspace topology are the
      homomorphisms whose values are specified at a finite number of points:

        \[\{h \in \Hom(A,\{0,1\}): h(i_1) = \epsilon_1, ... , h(i_n) = \epsilon_n\}\]

      So, every set defined as in the proposition is in the basis.

      Every set in the basis can be defined this way:

      Let $\Delta$ be an arbitrary element in the basis, given by \[\Delta = \{h
      \in S(A) : h(a_1) = \epsilon_1, ... , h(a_n) = \epsilon_n\}\]

      Where $a_i \in A$, and $\epsilon_i$ are either zero or one. Define:

      \[b_k = a_k \text{ if } \epsilon_k = 1, 1+a_k \text{ otherwise }\]

      Since the
      elements of $S(A)$ are homomorphisms, if $\epsilon_k = 1$,

      \[h(b_k) = h(a_k)\]

      and, if $\epsilon_k = 0$,

      \[h(b_k) = h(1+a_k) = 1+h(a_k)\]

      In other words, $h(b_k) = 1$ if and only if $h(a_k) = \epsilon_k$. Then,
      $h(a_1) = \epsilon_1, h(a_2) = \epsilon_2, ..., h(a_n) = \epsilon_n$ if and
      only if \[h(b_1) \frown h(b_2) \frown ... \frown h(b_n) = 1\] Since $h$
      also preserves intersections, this happens if and only if

      \[h(b_1 \frown b_2 \frown ... \frown b_n) = 1\]

      So we've shown that every open set has the form $\{h : h(a) = 1\}$ for some
      $a \in A$.

      Maybe as a special case, notice that the empty set and $S(A)$ are given by
      $\{h \in S(A) : h(0) = 1\}$ and $\{h \in S(A) : h(1) = 1\}$ respectively.

      Now, we'll show that the element $a$ is unique. Let $a \neq b$. Then, $a+b
      \neq 0$.

      Then, as we've shown, there's an ultrafilter that includes $a+b$, so a
      homomorphism with

      \[\phi(a+b) = 1 = \phi(a) + \phi(b)\]

      Since the right hand side is in $\{0,1\}$, this implies $\phi(a) = 1$ and
      $\phi(b) = 0$, or vice versa. Without loss of generality, assume $\phi(a)
      = 1$. Then,

      \[\phi \in \{h \in S(A): h(a) = 1\}\] \[\phi \notin \{h \in S(A) : h(b) =
      1\}\]

      So the sets aren't equal.
      \end{proof}

      As a corollary of this characterization of basis elements, the complements
      of the sets in the basis are still in the basis:

      \[\{h \in S(A) : h(a) = 1\}^c = \{h \in S(A) : h(1+a) = 1\}\]

      So the sets in the basis are clopen.  As the subspace of a Hausdorff
      space, $S(A)$ is Hausdorff.  If we can also show that it's compact, we'll
      have shown that it's a Boolean topological space.  Since the product space
      $\{0,1\}^A$ is compact by Tychonoff's theorem, we only need to show that
      it's closed, since a closed subset of a compact space is compact.

      We've proved that a function in $\{0,1\}^A$ is a homomorphism of Boolean
      algebras if and only if it preserves intersections and complements.

      Let \[\Omega(a,b) = \{ f: A \rightarrow \{0,1\} : f(ab) = f(a)f(b), f(1+a)
      = 1 + f(a)\}\]

      Then, the set of homomorphisms $S(A)$ is exactly:

      \[S(A) = \bigcap_{a \in A, b \in A} \Omega(a,b)\]

      We can each write $\Omega(a,b)$ as the following finite union of closed
      sets:

      \[\Omega(a,b) = \{f : f(a) = 0, f(b) = 0, f(ab) = 0), f(1+a) = 1\}\]
      \[\cup \{f : f(a) = 0, f(b) = 1, f(ab) = 0), f(1+a) = 1\}\] \[\cup \{f :
      f(a) = 1, f(b) = 1, f(ab) = 0), f(1+a) = 0\}\] \[\cup \{f : f(a) = 1, f(b)
      = 1, f(ab) = 1), f(1+a) = 0\}\]

      Then, each $\Omega(a,b)$ is closed as a finite union of closed sets, and
      $S(A)$ itself is closed, since it's some arbitrary intersection of closed
      sets.

      Then, it's a Boolean topological space.

      We already know that the sets in the basis are clopen, but we'll later
      need the fact that the clopen sets are exactly the sets in this basis.

      Given an open and closed set $\Gamma$,

      Since it's open, it's covered by sets in the basis $\{\Gamma_i\}_{i \in
      I}$ Since $\Gamma$ is a closed subset of a compact space, it's compact.
      Then, find a finite subcover $\Gamma_1, ... \Gamma_n$. Each has form:
      \[\Gamma_i = \{h: h(x_i) = 1\}\] Then, \[\Gamma = \{h: h(x_1 \smile x_2
      \smile ... \smile x_n) = 1\}\] and so it is in the basis.

    \section{Characterization of Boolean Algebras}

      Given a topological space $S$, the set of its clopen subsets form a
      Boolean algebra under the regular set operations union, intersection,
      complement -- clearly, the intersection or complement of clopen sets is
      clopen. This will be denoted as $B(S)$. Given a connected space, the only
      clopen sets are $\emptyset$ and $S$, so this Boolean algebra is isomorphic
      to $\{0,1\}$.

      We'll now prove that every Boolean algebra is isomorphic to the Boolean
      algebra of clopen sets of its Stone space.

      Let $H$ be the map $A \rightarrow \mathcal{P}(S(A))$

      \[H(a) = \{h \in S(A) : h(a) = 1\}\]

      We've done a lot of the work in the previous sections: We know that $H(a)$
      is a clopen subset of $S(A)$ for any $a$, so the codomain is actually
      $B(S(A))$.

      We know that any basis element, so any clopen subset, has this form, so
      the map is surjective.  We also know that given any clopen set, $a$ in the
      expression is unique, so the map is injective.

      To show that it's an isomorphism of Boolean algebras, we'll show that
      $H(a) \leq H(b)$ if and only if $a \leq b$.

      If $x \leq y$, and $h$ is any homomorphism with $h(x) = 1$, $h(x) \leq
      h(y)$ implies that $h(y) = 1$. Then, $H(x) \subseteq H(y)$.

      Now, assume $x$ is not less than or equal to $y$. Then, $xy \neq x$,
      $x(1+y) \neq 0$.  As we've remarked, for any element, we can find an
      ultrafilter that includes it, so a homomorphism $h$ with \[h(x(1+y)) = 1\]
      Then, \[h(x)(1+h(y)) = 1\] \[h(x) = 1, h(y) = 0\] And then, $h \in H(x)$,
      $h \notin  H(y)$, and $H(x) \nsubseteq H(y)$.

      \subsection{An Example}

        This result shows that every Boolean algebra $A$ is isomorphic to
        $B(S(A))$, so the Boolean algebra of certain subsets of some set. Can we
        say that every Boolean algebra is isomorphic to the Boolean algebra of
        the set of all subsets of some set? While, as we'll see, this is true
        for finite Boolean algebras, it's not true in the general case.

        To see why, define the following notion:

        \begin{definition}
          An atom in a Boolean algebra is an element $a$, such that if $x \leq
          a$, $x = a$ or $x = 0$.
        \end{definition}
        \begin{definition}
          A Boolean algebra is atomic, if for all nonzero $x$, there's an atom $a$ with
          $a \leq x$.
        \end{definition}

        Given the Boolean algebra of all subsets of some set, the singleton sets
        are atoms (if $A \subseteq \{a\}$, $A = \{a\}$ or $A = \emptyset$).

        Since every nonempty set has a singleton as a subset, all Boolean
        algebras of this form are atomic.

        We'll now construct a nonatomic Boolean algebra.

        Let $X$ be some infinite set, and $A$ be the set of formulas from
        propositional logic with variables from $X$, under the
        equivalence classes of logical equivalence.

        In propositional logic, the allowed operations are $\wedge$ and $\neg$,
        and other symbols like $\vee$ and $\rightarrow$ can be defined in
        terms of them. So, our objects are finite strings like

        \[ (\neg X_1 \vee X_2) \rightarrow X_3 \]

        where $X_1, X_2, X_3 \in X$.

        Given an assignment of truth variables, something like $X_1 = $ true,
        $X_2 = $ false, ... for every item of $X$, every formula evaluates to
        either true or false. We are considering elements under the equivalence
        classes where $F_1 \sim F_2$ if and only if they
        evaluate to the same value given every assignment of $X$. This is a
        Boolean algebra with $\frown = \wedge$, $\smile = \vee$, $' = \neg$, etc.

        Then, $\leq$, usually defined $F_1 \leq F_2$ iff $(F_1) \wedge (F_2) =
        F_1$, is equivalent to:

        $F_1 \leq F_2$ if and only if every assignment of $X$ that makes
        $F_1$ true also makes $F_2$ true.

        Under the equivalence classes, $F_1 \leq F_2$ and $F_2 \leq F_1$ implies
        $F_1 \sim F_2$. $0$ is the formula which is false for every assignment,
        and $1$ is the formula which is true for every assignment.

        Now, let $F$ be any nonzero formula. Since it's a finite string, pick a variable
        $x$ which is not in $F$. Form the formula

        \[G = x \wedge (F)\]

        Clearly, every assignment of variables that makes $G$ true also makes $F$
        true, so $G \leq F$.

        Since $F$ is nonzero, pick an assignment of variables that makes $F$
        true. Since $x$ isn't in $F$, it will still make $F$ true regardless of
        what it assigns to $x$. Let $\delta_0$ be a version which assigns false,
        and $\delta_1$ be a version that assigns true.

        Then, $G(\delta_1) = $ true, so $G \neq 0$, $G(\delta_0) = $ false,
        $F(\delta_0) = $ true, so $G < F$. Then, $F$ is not an atom.

        So, this Boolean algebra doesn't have any atoms. It can not be the
        Boolean algebra of all subsets of some set.

    \section{Characterization of Boolean Topological Spaces}

      We can also now prove that every Boolean topological space $X$ is
      homeomorphic to the Stone space of the Boolean algebra of clopen subsets
      of $X$, so $X = S(B(X))$.

      Take the element  $x \in X$ into:

      \[f_x: B(X) \rightarrow \{0,1\} = \Omega \mapsto \{1 \text { if } x \in
      \Omega, 0 \text { otherwise }\}\]

      First, we'll show that $f_x$ is really a homomorphism of Boolean algebras,
      so in $S(B(X))$.

      For clopen subsets $\Omega, \Delta$,

      \[f_x(\Omega \cap \Delta) = 1 \text{ iff } x \in \Omega \cap \Delta\]
      \[= 1 \text{ iff } f_x(\Omega) = 1 \text{ and } f_x(\Delta) = 1\]
      \[= f_x(\Omega)f_x(\Delta)\]

      \[f_x(\Omega^c) = 1 \text{ iff } x \in \Omega^c\]
      \[ = 0 \text{ iff } x \in \Omega\]
      \[ = f_x(\Omega)^c\]

      So, $f_x$ is a Boolean algebra homomorphism from $B(X)$ onto $\{0,1\}$, so
      is in $S(B(X))$.

      Now, we'll show that the mapping $x \mapsto f_x$ is injective.

      If $x \neq y$, because $X$ is Hausdorff, we can find an open set $O$ with
      $x \in O$, $y \notin O$. $O$ is a union of basic open sets, so there's a
      clopen set $\Omega \in B(X)$ with $x \in \Omega$, $y\notin \Omega$. Then,

      \[f_x(\Omega) = 1, f_y(\Omega) = 0\]

      So, $f_x \neq f_y$, and $f$ is injective.

      Now, we'll show that it's surjective.

      Let $h$ be an arbitrary element of $S(B(X))$, so a homomorphism $B(X)
      \rightarrow \{0,1\}$. The ultrafilter associated with $h$ is

      \[U_h = \{\Omega \in B(X) : h(\Omega) = 1\}\]

      Any ultrafilter on a compact and Hausdorff space converges to a unique
      point.

      Let $x$ be this element the ultrafilter of $h$ converges to. Since every
      element $\Omega$ of $B(X)$ is closed, $x$ is also the unique element of
      the intersection of all the elements of $h$.

      For every clopen subset $\Omega$ of $B(X)$,

      $\Omega$ might be an element of $U$. In this case, since $x$ is in every
      element of $U$, it's in $\Omega$, and

      \[f_x(\Omega) = 1\]

      Otherwise, $\Omega$ isn't an element of $U$, Then, $1+\Omega$ is, since
      $U$ is an ultrafilter. Then, $x$ is in $1+\Omega$, and

      \[f_x(1+\Omega) = 1\]

      And since $f_x$ is a homomorphism,

      \[f_x(\Omega) = 0\]

      So we have that

      \[f_x(\Omega) = 1 \text{ iff } \Omega \in U_h\]

      And, by definition, $U_h = \{\Omega \in B(X) : h(\Omega) = 1\}$. So,

      \[f_x = h\]

      And the mapping is surjective.

      Finally, we can prove that $f$ is continuous:

      Let $G$ be an element in the basis of $S(B(X))$, so a clopen set. We've
      seen that in Stone spaces, this means it has the form

      \[G = \{h \in S(B(X)): h(\Omega) = 1\}\]

      For some $\Omega \in B(X)$.

      \[f^{-1}(G) = \{x \in X : f_x \in G\}\]
      \[ = \{x \in X : f_x(\Omega) = 1\}\]
      \[ = \{x \in X : x \in \Omega\}\]
      \[ = \Omega\]

      Where $\Omega \in B(X)$, so it's open. Since the inverse image of elements
      in the basis are open, $f$ is continuous.

      $f^{-1}$ is continuous:

      Let $\Omega \in X$ be an open set in the basis of $X$.

      \[(f^{-1})^{-1}(\Omega) = f(\Omega) = \{f_x \in S(B(X)) : x \in \Omega\}\]

      We've seen that $f$ is surjective, so every $h \in S(B(X))$ has form $f_x$
      for some $x$. Then, $h(\Omega) = 1$ if and only if $x$ such that $h = f_x$
      is in $\Omega$ -- by definition of $f$, $f_x(\Omega) = 1$ if and only if
      $x$ is in $\Omega$. This means:

      \[\{f_x \in S(B(X)) : x \in \Omega\} = \{h \in S(B(X)) : h(\Omega) = 1\}\]

      The latter is a basis element in the Stone space of $B(X)$, so we conclude
      that $f^{-1}$ is continuous as well. So, $x \mapsto f_x: X \rightarrow
      S(B(X))$ is a homeomorphism of topological spaces.

      \subsection{An Example}

        Let $X$ be the set $\{a,b,c\}$ with the discrete topology. This space is
        clearly Hausdorff, compact and zero dimensional.

        $B(X)$ is the Boolean algebra with eight elements of $\mathcal{P}(X)$
        under the usual set operations.

        The underlying set of $S(B(X))$ is homomorphisms $B(X) \to \{0,1\}$.

        Let $f$ be such a function. Notice that out of $f(\{a\})$, $f(\{b\})$,
        $f(\{c\})$, at most one can be true, since, for example

        \[f(\{a\}) = f(\{b\} \cup \{c\})^c = (f(\{b\}) \vee f(\{c\}))^c\]

        Also, since $f(1) = f(\{a\}) \vee f(\{b\}) \vee f(\{c\}) = 1$, exactly
        one of these must be true. So, $S(B(X))$ has three elements:

        \[f_{\{a\}} = \{a\} \mapsto 1, \{b\} \mapsto 0, \{c\} \mapsto 0\]
        \[f_{\{b\}} = \{a\} \mapsto 0, \{b\} \mapsto 1, \{c\} \mapsto 0\]
        \[f_{\{c\}} = \{a\} \mapsto 0, \{b\} \mapsto 0, \{c\} \mapsto 1\]

        Where each of the functions are extended to $\mathcal{P}(X)$.
        Since $\{0,1\}^3$ has the discrete topology, $S(B(X))$ has the discrete
        topology too. Clearly, $f$ is a homeomorphism.

        Since every finite Hausdorff topology is discrete, this is the most
        interesting finite example we could come up with.

        If we started from a finite Boolean algebra $A$ instead, for example
        $\{\{1,2\},\{3\}\}$ under the set operations, $S(A)$ would still be a
        finite Hausdorff space, therefore have the discrete topology, then
        $B(S(A))$ would be the set of all subsets of $S(A)$. So, every finite
        Boolean algebra is isomorphic to $\mathcal{P}(\{1, ..., n\})$ for some
        finite set $\{1, ..., n\}$, and has $2^n$ elements.

    \section{Categorical Equivalence}

      In this section, $H_A$ is the map $A \to B(S(A))$ we found two sections
      earlier.

      We will now construct a bijection between homomorphisms between Boolean
      algebras $A$ to $A'$ and continuous functions $S(A')$ to $S(A)$.

      Let $\phi: A \to A'$ be a homomorphism.  \[\Phi(\phi) = (h : A'
      \rightarrow \{0,1\}) \mapsto (h \circ \phi : A \rightarrow \{0,1\})\]

      We'll first show that this map is actually continuous.

      Let $\Omega$ be an element in the basis of $S(A)$. It has form $\{h \in
      S(A): h(a) = 1\}$ for some $a \in A$.

      \[(\Phi(\phi))^{-1}(\Omega) = \left\{h' \in S(A') : \Phi(\phi)(h') \in \{h
      \in S(A): h(a) = 1\} \right\}\] \[ = \{h' \in S(A') : (\Phi(\phi)(h'))(a)
      = 1 \}\] \[ = \{h' \in S(A') : (h' \circ \phi)(a) = 1 \}\] \[ = \{h' \in
      S(A') : h'(\phi(a)) = 1\}\]

      This is an open set in $S(A')$, so this map is continuous.

      We'll show that this is a bijection by defining an inverse. If $\alpha$ is
      a continuous function $S(A')$ to $S(A)$, let $\alpha^{-1}$  be the
      function $\mathcal{P}(S(A)) \to \mathcal{P}(S(A'))$. Since $\alpha$ is
      continuous, $\alpha^{-1}$ takes clopen sets to clopen sets, and so it's a
      map $B(S(A')) \to B(S(A))$. Then, the following defines a function $A \to
      B(S(A)) \to B(S(A')) \to A'$:

      \[\Psi(\alpha) = H_{A'}^{-1} \circ \alpha^{-1} \circ H_A\]

      \[(\Psi \circ \Phi)(\phi) = \Psi(\Phi(\phi))\]
      \[ = H_{A'}^{-1} \circ (\Phi(\phi))^{-1} \circ H_A \]
      \[ = a \mapsto (H_{A'}^{-1} \circ (\Phi(\phi))^{-1}) \left(\{h \in S(A) :
      h(a) = 1 \}\right) \]
      \[ = a \mapsto H_{A'}^{-1} \left( \left\{g \in S(A') : \Phi(\phi)(g) \in
      \{h \in S(A) : h(a) = 1 \}\right\} \right) \]
      \[ = a \mapsto H_{A'}^{-1} (\{g \in S(A') : ((\Phi(\phi))(g))(a) = 1 \})\]
      \[ = a \mapsto H_{A'}^{-1} (\{g \in S(A') : ((g \circ \phi)(a) = 1 \})) \]
      \[ = a \mapsto H_{A'}^{-1} (\{g \in S(A') : g(\phi(a)) = 1 \}) \]
      \[ = a \mapsto H_{A'}^{-1} (H_{A'}(\phi(a))) \]
      \[ = a \mapsto \phi(a)\]
      \[ = \phi\]

      \[(\Phi \circ \Psi)(\alpha) = \Phi(H_{A'}^{-1} \circ \alpha^{-1} \circ
      H_A)\]
      \[= h \mapsto h \circ H_{A'}^{-1} \circ \alpha^{-1} \circ H_A \]
      \[= h \mapsto (a \mapsto (h \circ H_{A'}^{-1} \circ \alpha^{-1} \circ
      H_A)(a)) \]

      Let $U$ be the argument of $h \circ H_{A'}^{-1}$ in the expression, so

      \[U \in B(S(A)) = \alpha^{-1} (H_A (a))\]
      \[ = \alpha^{-1} \{h \in S(A) : h(a) = 1 \}\]
      \[ = \{h' \in S(A') :\alpha(h') \in \{ h \in S(A) : h(a) = 1\} \}\]
      \[ = \{h' \in S(A') : \alpha(h')(a) = 1\} \]


      $U$ is equal to $\{h \in S(A') : h(u) = 1\}$ for some $u \in A'$. Then,

      \[ h ( H_{A'}^{-1} (U)) =  h(H_A^{-1}(\{h \in S(X) : h(u) = 1\})) =
      h(H_A^{-1}(H_A(u))) = h(u)\]

      By definition, $u$ is the element of $A'$ such that $h \in U$ if and only
      if $h(u) = 1$. Then, \[ h ( H_{A'}^{-1} (U)) = 1 \text{ if and only if } h
      \in U\] \[ h ( H_{A'}^{-1} (U)) = 1 \text{ if and only if } h \in \{h'
      \in S(A') : \alpha(h')(a) = 1\} \] \[ h ( H_{A'}^{-1} (U)) = 1 \text{ if
      and only if } \alpha(h)(a) = 1\] Since all the possible values are $0$
      or $1$,

      \[ h ( H_{A'}^{-1} (U)) = \alpha(h)(a)\] \[(\Phi \circ \Psi)(\alpha) = h
      \mapsto (a \mapsto \alpha(h)(a))\] \[(\Phi \circ \Psi)(\alpha) =  h
      \mapsto \alpha(h)\] \[(\Phi \circ \Psi)(\alpha) = \alpha\]

      So, $\Phi$ defines a bijective map from the morphisms of the category of Boolean
      algebras to the morphisms of the category of Boolean
      topological spaces and continuous functions. With $S$ mapping the objects,
      it's a contravariant functor:

      \[ \Phi(\text{id}) = h \mapsto h \circ \text{id} = \text{id} \]
      \[ \Phi(a \circ b) = h \mapsto h \circ (a \circ b) = (g \mapsto g \circ b)
      \circ (h \mapsto h \circ a) = \Phi(b) \circ \Phi(a) \]

      Similarly $\Psi$ defines a functor mapping the objects $S(X)$ to $X$ (not
      $B(S(X))$):

      \[ \Psi(\text{id}) = H_A^{-1} \circ \text{id}^{-1} \circ H_A = \text{id} \]
      \[ \Psi(a \circ b) = H_C^{-1} \circ (a \circ b)^{-1} \circ H_A \]
      \[ = H_C^{-1} \circ b^{-1} \circ a^{-1} \circ H_A \]
      \[ = (H_C^{-1} \circ b^{-1} \circ H_B) \circ (H_B^{-1} \circ a^{-1} \circ
      H_A) \]
      \[ = \Psi(b) \circ \Psi(a) \]

      Since $\Psi \circ \Phi$ maps objects to themselves and $\Phi \circ \Psi$
      maps objects to isomorphic objects, they're natural transformations of the
      identity functors, and this forms an equivalence of categories.

    \clearpage

    \begin{thebibliography}{9}

      \bibitem{bool} Cori Ren\'e, and Daniel Lascar. “Chapter 2: Boolean
      Algebras.” Mathematical Logic: A Course with Exercises, Oxford University
      Press, Oxford, 2008. 

      \bibitem{zorn} Lewin, Jonathan. “A Simple Proof of Zorn's Lemma.” The
      American Mathematical Monthly, vol. 98, no. 4, 1991, p. 353.,
      \url{https://doi.org/10.2307/2323807}.

      \bibitem{tych} Norwood, Zack. FILTER CONVERGENCE AND TYCHONOFF’S THEOREM.
      \url{http://www-personal.umich.edu/~norwoodz/ucla/files/tychultrafilters.pdf}.

      \bibitem{wiki} “Stone's Representation Theorem for Boolean Algebras.”
      Wikipedia, Wikimedia Foundation, 29 Jan. 2023,
      \url{https://en.wikipedia.org/wiki/Stone%27s_representation_theorem_for_Boolean_algebras}.

    \end{thebibliography}


\end{document}

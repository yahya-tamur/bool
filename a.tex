\documentclass{article}

\usepackage{amsmath}
% \usepackage{fourier}

\renewcommand{\thesubsection}{\thesection.\alph{subsection}}

\title{}
\author{}
\date{}

\begin{document}
  \maketitle
  \section{}

    Let $B$ be a ring where $x = x^2$ for all $x$. Some immediate consequences
    are:

    $B$ is commutative:

    Every element is its own additive inverse:

    \[ x + x = (x + x)^2 = x^2 + x^2 + x^2 + x^2 = x + x + x + x\]
    \[0 = x + x\]

    Multiplication is commutative:

    \[ x + y = (x + y)^2 = x^2 + xy + yx + y^2 = x + xy + yx + y\]
    \[0 = xy + yx\]
    \[xy = -yx = yx\]

    Where the last identity follows from our previous result.

    Now, define the following binary relation: $x \leq y$ iff $xy = x$. We get:

    \[xx = x \text{, so } x \leq x\]
    \[xy = x, yz = y \rightarrow xz = x(yz) = (xy)z = yz = x \text{, so } x
    \leq y, y \leq z \rightarrow x \leq z\]

    \[xy = x, yx = y \rightarrow x = xy = yx = y \text{, so } x \leq y, y \leq
    x \rightarrow x = y\]

    So, this is a reflexive partial order. Furthermore,

    \[0x = 0, x1 = x \rightarrow 0 \leq x, x \leq 1\]

    So we have greatest and least elements. We also have least upper bounds and
    greatest upper bounds:

    If $c \leq x, c \leq y$, then $c \leq xy \leq x, y$: By definition,
    $cx = c, cy = c$. Then,

    \[ cxy = cy = c \rightarrow c \leq xy\]
    \[ x(xy) = xy \rightarrow xy \leq x\]
    \[ y(xy) = (xy)y = xy \rightarrow xy \leq y\]


    If $x \leq c, y \leq c$, $x, y \leq x + xy + y \leq c$: By definition,
    $xc = x, yc = y$. Then,

    \[(x + y + xy)c = xc + yc + xyc = x + y + xy \rightarrow x + y + xy \leq c\]
    \[(x + y + xy)x = x + yx + yx = x \rightarrow x \leq x + y + xy\]
    \[(x + y + xy)y = xy + y + xy = y \rightarrow y \leq x + y + xy\]

    A partial order with least upper bounds ('joins') and greatest lower bounds
    ('meets') is called a lattice; a lattice with a greatest and least element
    is called a bounded latice. (wikipedia)

    Denoting the operations $x \frown y = xy, x \smile y = x + y + xy$, these
    operations a

    This is object is called a boolean algebra. We will now proceed to define
    them by starting from a lattice and defining a ring structure. One advantage
    of that approach is that in order to verify a map between boolean algebras
    is a homomorphism, it will suffice to check that it respects the order
    relation ($x \leq y$ iff $f(x) \leq f(y)$).





\end{document}

